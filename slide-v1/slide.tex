\documentclass[10pt]{beamer}

\usetheme{metropolis}
\usepackage{appendixnumberbeamer}
\usepackage[utf8]{vietnam}
\usepackage{booktabs}
\usepackage[scale=2]{ccicons}

\usepackage{pgfplots}
\usepgfplotslibrary{dateplot}

\usepackage{xspace}
\newcommand{\themename}{\textbf{\textsc{metropolis}}\xspace}

\title{Đề cương luận văn thạc sĩ}
\subtitle{Nghiên cứu phát triển kỹ thuật đếm số phần tử \\ \hspace{2.4cm} trên dòng dữ liệu}
\date{Người hướng dẫn khoa học: \hspace{0.5cm}PGS. TS. THOẠI NAM}
\author{Học viên: Lê Anh Quốc \hspace{1cm} ID: 2070428}
%\institute{\today}
\titlegraphic{\hfill\includegraphics[height=1.5cm]{hcmut.png}}

\begin{document}

\maketitle

\begin{frame}{Outline}
  \setbeamertemplate{section in toc}[sections numbered]
  \tableofcontents[hideallsubsections]
\end{frame}

\section{Giới thiệu}

\begin{frame}[fragile]{Giới thiệu}
Ngày nay, các ứng dụng và dịch vụ trực tuyến đóng vai trò ngày càng quan trọng trong cuộc sống của
con người. Chúng ta sử dụng mạng xã hội để kết nối với bạn bè và chia sẻ thông tin, mua sắm trực tuyến
để tiết kiệm thời gian và tiền bạc, hay xem phim và chơi game trực tuyến để giải trí. Để đánh giá hiệu quả
hoạt động của các ứng dụng và dịch vụ này, một trọng những chỉ số quan trọng nhất là số lượng người dùng
hoạt động.
Việc theo dõi số lượng người dùng hoạt động trong một khoảng thời gian nhất định trên một dòng dữ
liệu (data stream) là một yêu cầu quan trọng đối với nhiều ứng dụng và dịch vụ trực tuyến, hiệu quả của
các chiến dịch marketing, và hỗ trợ ra quyết định kinh doanh.
Ví dụ, trong các ứng dụng mạng xã hội, số lượng người dùng hoạt động cho thấy mức độ tương tác và sự
quan tâm của người dùng đối với nền tảng. Trong các dịch vụ thương mại điện tử, số lượng người dùng hoạt
động cho thấy hiệu quả và các chiến dịch quảng cáo và khuyến mãi.
Tuy nhiên, việc đếm số lượng người dùng không phải là một nhiệm vụ đơn giản, đặc biệt là khi dữ liệu lớn
và tốc độ truy cập cao. Các phương pháp truyền thống như lưu trữ và truy vấn trực tiếp vào cơ sở dữ liệu
có thể gặp nhiều hạn chế về hiệu suất và khả năng mở rộng.
\end{frame}
\begin{frame}[fragile]{Giới thiệu}
  Trong nhiều trường hợp, cần phải tổng hợp số lượng người dùng trên nhiều dòng dữ liệu khác nhau. Việc
này giúp có được bức tranh toàn cảnh về hoạt động của người dùng trên toàn hệ thống, từ đó đưa ra các
phân tích và đánh giá chính xác hơn.
Ví dụ, trong hệ thống thương mại điện tử, cần tổng hợp số lượng người dùng từ các trang web, ứng dụng di
động và API khác nhau để có được số lượng người dùng hoạt động thực tế trên toàn hệ thống. Tuy nhiên,
việc tổng hợp dữ liệu từ nhiều nguồn khác nhau có thể gặp thách thức về đồng bộ hóa dữ liệu, xử lý dữ liệu
bị thiếu hoặc lỗi, và đảm bảo tính nhất quán của kết quả.
Ngoài ra, có thể cần phải đếm số lượng người dùng trên nhiều khoảng thời gian khác nhau trên một hoặc
nhiều dòng dữ liệu khác nhau. Việc này giúp phân tích chi tiết hơn hoạt động của người dùng theo thời gian,
theo khu vực hoặc theo tiêu chí khác.
\end{frame}

\begin{frame}[fragile]{Giới thiệu}
Ví dụ, trong một ứng dụng phát trực tiếp, cần đếm số lượng người dùng hoạt động theo giờ hoặc từng phân
đoạn chương trình để đánh giá mức độ quan tâm của người xem. Tuy nhiên, việc phân chia và xử lý dữ liệu
theo nhiều đoạn có thể làm tăng độ phức tạp của thuật toán và ảnh hưởng đến hiệu suất của hệ thống. Do
đó, cần phải có một giải pháp đếm số lượng phần tử trên dòng dữ liệu đạt hiệu suất cao và tin cậy, từ đó
có thể ứng dụng rộng rãi trong các hệ thống khác nhau như mạng xã hội, thương mại điện tử, chương trình
phát trực tiếp, hệ thống giám sát và hệ thống giao thông thông minh.
\end{frame}

\begin{frame}[fragile]{Sections}
  Sections group slides of the same topic

  \begin{verbatim}    \section{Elements}\end{verbatim}

  for which \themename provides a nice progress indicator \ldots
\end{frame}
\section{Các công trình nghiên cứu liên quan}
\begin{frame}[fragile]{LogLog \cite{durand2003loglog}}
  Thuật toán LogLog cho phép ước lượng số lượng từ vựng khác nhau trong toàn bộ tác phẩm
  của Shakespeare chỉ trong một lần quét và với độ chính xác cỡ vài phần trăm, sử dụng một lượng bộ nhớ
  phụ nhỏ. Phiên bản cơ bản đã được xác minh qua phân tích toàn diện và có phiên bản tối ưu hóa có khả
  năng song song.
\end{frame}
\begin{frame}[fragile]{HyperLogLog \cite{flajolet2007hyperloglog}}
  Thuật toán HYPERLOGLOG là một thuật toán xác suất gần tối ưu, được thiết kế để ước lượng số lượng các phần tử khác nhau trong các tập dữ liệu rất lớn. 
  Sử dụng bộ nhớ phụ có kích thước m đơn vị, HYPERLOGLOG thực hiện một lần quét qua dữ liệu và tạo ra một ước lượng về số lượng phần tử khác nhau với độ chính xác tương đối là khoảng $\frac{1.04}{\sqrt{m}}$. 
  Thuật toán này có khả năng ước lượng số lượng phần tử lớn hơn $10^9$ với độ chính xác khoảng 2\% chỉ sử dụng 1.5 kilobytes bộ nhớ, đồng thời có khả năng song song hoá tối ưu 
  và thích nghi với mô hình cửa sổ trượt (sliding windown).
\end{frame}
\begin{frame}[fragile]{HyperLogLog++ \cite{heule2013hyperloglog}}
  Bài báo giới thiệu một thuật toán mới ước lượng số lượng luồng hoạt động trong dòng dữ liệu, sử dụng cơ chế cửa sổ trượt kết hợp với 
  thuật toán HyperLogLog. Thuật toán này có độ chính xác cao, lỗi tiêu chuẩn khoảng $\frac{1.04}{\sqrt{m}}$, với m là số lượng thanh ghi trong bộ nhớ. Dù cần bộ nhớ bổ sung 
  so với HyperLogLog, tổng bộ nhớ cần thiết không vượt quá $5m\ln(\frac{n}{m})$ byte, với n là số luồng thực sự trong cửa sổ trượt. Kết quả lý thuyết được xác minh trên cả dữ liệu thực 
  và tổng hợp.
\end{frame}
\begin{frame}[fragile]{Sliding HyperLogLog \cite{chabchoub2010sliding}}
  Bài báo giới thiệu một thuật toán mới ước lượng số lượng luồng hoạt động trong dòng dữ liệu, sử dụng cơ chế cửa sổ trượt kết hợp với 
  thuật toán HyperLogLog. Thuật toán này có độ chính xác cao, lỗi tiêu chuẩn khoảng $\frac{1.04}{\sqrt{m}}$, với m là số lượng thanh ghi trong bộ nhớ. Dù cần bộ nhớ bổ sung 
  so với HyperLogLog, tổng bộ nhớ cần thiết không vượt quá $5m\ln(\frac{n}{m})$ byte, với n là số luồng thực sự trong cửa sổ trượt. Kết quả lý thuyết được xác minh trên cả dữ liệu thực 
  và tổng hợp.
\end{frame}
\begin{frame}[fragile]{ExaLogLog \cite{ertl2024exaloglog}}
  ExaLogLog là một cấu trúc dữ liệu mới cho việc đếm độc lập xấp xỉ, tương tự như HyperLogLog, nhưng tiêu tốn ít hơn 43\% không gian với cùng lỗi ước lượng.
\end{frame}
\section{Phát biểu bài toán}
\begin{frame}[fragile]{Phát biểu bài toán}
\textbf{Bài toán 1:} Phát triển thuật toán để ước lượng số lượng phần tử (cardinality estimation) trong một khoảng thời gian trên một dòng dữ liệu (data stream).\\
\textbf{Bài toán 2:} Mở rộng thuật toán để ước lượng số lượng phần tử trong một khoảng thời gian trên nhiều dòng dữ liệu.
\end{frame}
\section{Mục tiêu, đối tượng và giới hạn nghiên cứu}

\begin{frame}{Mục tiêu}
	Mục tiêu chính:
  \begin{itemize}
    \item Phát triển kỹ thuật đếm số lượng phần tử hiệu quả, có chính xác cao trên dòng dữ liệu.
    \item Nâng cao hiệu suất xử lý dữ liệu lớn, đáp ứng nhu cầu ngày càng tăng trong kỷ nguyên số.
    \item Đóng góp vào sự phát triển của công nghệ dữ liệu lớn, mở ra tiềm năng ứng dụng rộng lớn trong nhiều lĩnh vực.
  \end{itemize}
Mục tiêu cụ thể:
  \begin{itemize}
    \item Phân tích và đánh giá các kỹ thuật đếm số lượng phần tử hiện có.
    \item Đề xuất và triển khai kỹ thuật đếm số lượng phần tử mới, tối ưu hóa hiệu suất và độ chính xác.
    \item Thực hiện thí nghiệm để chứng minh tính ưu việt của kỹ thuật mới so với các kỹ thuật hiện có.
    \item Phân tích kết quả thí nghiệm, rút ra kết luận và đề xuất hướng nghiên cứu tiếp theo.
  \end{itemize}
\end{frame}
\begin{frame}{Giới hạn}
  \begin{itemize}
      \item Đề tài tập trung nghiên cứu kỹ thuật đếm số lượng phần tử trên dòng dữ liệu dạng văn bản.
      \item Các kỹ thuật được đề xuất và triển khai có thể chưa áp dụng được cho tất cả các loại dữ liệu.
      \item Nghiên cứu chỉ giới hạn trong thời gian cho phép.
  \end{itemize}
\end{frame}
\begin{frame}{Đối tượng nghiên cứu}
  Đối tượng nghiên cứu của đề tài "Nghiên cứu phát triển kỹ thuật đếm số lượng phần tử trên dòng dữ liệu"
  Đối tượng nghiên cứu chính:
  \begin{itemize}
      \item Dòng dữ liệu dạng văn bản có chứa nhiều phần tử cần đếm.
      \item Các kỹ thuật đếm số lượng phần tử hiện có và mới được đề xuất.
  \end{itemize}
  Đối tượng nghiên cứu cụ thể:
  \begin{itemize}
      \item Số phần tử có thể là userID, IP address hoặc bất kỳ đối tượng nào tương đương mà có thể đếm được nhờ định danh của nó.
      \item Các ứng dụng xử lý dữ liệu lớn có nhu cầu đếm số lượng phần tử hiệu quả.
  \end{itemize}
\end{frame}

\section{Cơ sở lý thuyết}

\begin{frame}[fragile]{Phát biểu bài toán}
      \begin{verbatim}The theme provides sensible defaults to
\emph{emphasize} text, \alert{accent} parts
or show \textbf{bold} results.\end{verbatim}

  \begin{center}becomes\end{center}

  The theme provides sensible defaults to \emph{emphasize} text,
  \alert{accent} parts or show \textbf{bold} results.
\end{frame}

\begin{frame}{Font feature test}
  \begin{itemize}
    \item Regular
    \item \textit{Italic}
    \item \textsc{Small Caps}
    \item \textbf{Bold}
    \item \textbf{\textit{Bold Italic}}
    \item \textbf{\textsc{Bold Small Caps}}
    \item \texttt{Monospace}
    \item \texttt{\textit{Monospace Italic}}
    \item \texttt{\textbf{Monospace Bold}}
    \item \texttt{\textbf{\textit{Monospace Bold Italic}}}
  \end{itemize}
\end{frame}

\begin{frame}{Lists}
  \begin{columns}[T,onlytextwidth]
    \column{0.33\textwidth}
      Items
      \begin{itemize}
        \item Milk \item Eggs \item Potatoes
      \end{itemize}

    \column{0.33\textwidth}
      Enumerations
      \begin{enumerate}
        \item First, \item Second and \item Last.
      \end{enumerate}

    \column{0.33\textwidth}
      Descriptions
      \begin{description}
        \item[PowerPoint] Meeh. \item[Beamer] Yeeeha.
      \end{description}
  \end{columns}
\end{frame}
\begin{frame}{Animation}
  \begin{itemize}[<+- | alert@+>]
    \item \alert<4>{This is\only<4>{ really} important}
    \item Now this
    \item And now this
  \end{itemize}
\end{frame}
\begin{frame}{Figures}
  \begin{figure}
    \newcounter{density}
    \setcounter{density}{20}
    \begin{tikzpicture}
      \def\couleur{alerted text.fg}
      \path[coordinate] (0,0)  coordinate(A)
                  ++( 90:5cm) coordinate(B)
                  ++(0:5cm) coordinate(C)
                  ++(-90:5cm) coordinate(D);
      \draw[fill=\couleur!\thedensity] (A) -- (B) -- (C) --(D) -- cycle;
      \foreach \x in {1,...,40}{%
          \pgfmathsetcounter{density}{\thedensity+20}
          \setcounter{density}{\thedensity}
          \path[coordinate] coordinate(X) at (A){};
          \path[coordinate] (A) -- (B) coordinate[pos=.10](A)
                              -- (C) coordinate[pos=.10](B)
                              -- (D) coordinate[pos=.10](C)
                              -- (X) coordinate[pos=.10](D);
          \draw[fill=\couleur!\thedensity] (A)--(B)--(C)-- (D) -- cycle;
      }
    \end{tikzpicture}
    \caption{Rotated square from
    \href{http://www.texample.net/tikz/examples/rotated-polygons/}{texample.net}.}
  \end{figure}
\end{frame}
\begin{frame}{Tables}
  \begin{table}
    \caption{Largest cities in the world (source: Wikipedia)}
    \begin{tabular}{@{} lr @{}}
      \toprule
      City & Population\\
      \midrule
      Mexico City & 20,116,842\\
      Shanghai & 19,210,000\\
      Peking & 15,796,450\\
      Istanbul & 14,160,467\\
      \bottomrule
    \end{tabular}
  \end{table}
\end{frame}
\begin{frame}{Blocks}
  Three different block environments are pre-defined and may be styled with an
  optional background color.

  \begin{columns}[T,onlytextwidth]
    \column{0.5\textwidth}
      \begin{block}{Default}
        Block content.
      \end{block}

      \begin{alertblock}{Alert}
        Block content.
      \end{alertblock}

      \begin{exampleblock}{Example}
        Block content.
      \end{exampleblock}

    \column{0.5\textwidth}

      \metroset{block=fill}

      \begin{block}{Default}
        Block content.
      \end{block}

      \begin{alertblock}{Alert}
        Block content.
      \end{alertblock}

      \begin{exampleblock}{Example}
        Block content.
      \end{exampleblock}

  \end{columns}
\end{frame}
\begin{frame}{Math}
  \begin{equation*}
    e = \lim_{n\to \infty} \left(1 + \frac{1}{n}\right)^n
  \end{equation*}
\end{frame}
\begin{frame}{Line plots}
  \begin{figure}
    \begin{tikzpicture}
      \begin{axis}[
        mlineplot,
        width=0.9\textwidth,
        height=6cm,
      ]

        \addplot {sin(deg(x))};
        \addplot+[samples=100] {sin(deg(2*x))};

      \end{axis}
    \end{tikzpicture}
  \end{figure}
\end{frame}
\begin{frame}{Bar charts}
  \begin{figure}
    \begin{tikzpicture}
      \begin{axis}[
        mbarplot,
        xlabel={Foo},
        ylabel={Bar},
        width=0.9\textwidth,
        height=6cm,
      ]

      \addplot plot coordinates {(1, 20) (2, 25) (3, 22.4) (4, 12.4)};
      \addplot plot coordinates {(1, 18) (2, 24) (3, 23.5) (4, 13.2)};
      \addplot plot coordinates {(1, 10) (2, 19) (3, 25) (4, 15.2)};

      \legend{lorem, ipsum, dolor}

      \end{axis}
    \end{tikzpicture}
  \end{figure}
\end{frame}
\begin{frame}{Quotes}
  \begin{quote}
    Veni, Vidi, Vici
  \end{quote}
\end{frame}

{%
\setbeamertemplate{frame footer}{My custom footer}
\begin{frame}[fragile]{Frame footer}
    \themename defines a custom beamer template to add a text to the footer. It can be set via
    \begin{verbatim}\setbeamertemplate{frame footer}{My custom footer}\end{verbatim}
\end{frame}
}

\begin{frame}{References}
  Some references to showcase [allowframebreaks] \cite{durand2003loglog, flajolet2007hyperloglog, chabchoub2010sliding, ertl2024exaloglog, heule2013hyperloglog}
\end{frame}

\section{Kiến thức nền tảng}

\begin{frame}{Kiến thức nền tảng}

  Get the source of this theme and the demo presentation from

  \begin{center}\url{github.com/matze/mtheme}\end{center}

  The theme \emph{itself} is licensed under a
  \href{http://creativecommons.org/licenses/by-sa/4.0/}{Creative Commons
  Attribution-ShareAlike 4.0 International License}.

  \begin{center}\ccbysa\end{center}

\end{frame}

\section{Phương pháp thực hiện}

\begin{frame}{Bài toán 1}
  \textbf{Bài toán 1:} Phát triển thuật toán để ước lượng số lượng phần tử (cardinality estimation) trong một khoảng thời gian trên một dòng dữ liệu (data stream):\\
  Trong phương pháp này, chúng tôi sẽ trình bày quá trình phát triển thuật toán sử dụng HyperLogLog để ước lượng 
  số lượng phần tử trên một dòng dữ liệu. Bằng cách sử dụng cấu trúc dữ liệu HyperLogLog thep khung thời gian, ví dụ như mỗi giờ hoặc mỗi phút và các kỹ thuật tối ưu, 
  chúng tôi xây dựng một thuật toán hiệu quả và chính xác để đếm số lượng phần tử duy nhất trong dữ liệu dòng.
  \begin{itemize}
    \item Bước 1: Xác định khoảng thời gian
    Đầu tiên, chúng tôi sẽ xác định khoảng thời gian mà chúng tôi muốn đếm số lượng phần tử. Ví dụ, mỗi giờ hoặc mỗi phút.
    \item Bước 2: Lưu trữ HyperLogLog
    Tiếp theo, chúng tôi sẽ lưu trữ cấu trúc HyperLogLog cho mỗi khoảng thời gian. Cấu trúc dữ liệu sẽ bao gồm cặp $\left< T_1, HLL_1\right>$, 
    trong đó $T_1$ là thời điểm đại diện cho khung thời gian cụ thể.
    \item Bước 3: Sử dụng kết quả
    Cuối cùng, khi cần, chúng tôi có thể truy vấn và sử dụng kết quả từ các cấu trúc HyperLogLog lưu trữ theo khung thời gian để 
    ước lượng số lượng phần tử trong mỗi khoảng thời gian.
\end{itemize}
\end{frame}

\begin{frame}{Bài toán 2}
  \textbf{Bài toán 2:} Mở rộng thuật toán để ước lượng số lượng phần tử trong một khoảng thời gian trên nhiều dòng dữ liệu:
  Trong phương pháp này, chúng tôi sẽ sẽ mở rộng thuật toán 1 để ước lượng trên nhiều dòng dữ liệu. Ví dụ khi chúng ta cần biết có bao nhiêu người dùng
  đã đăng nhập vào hệ thống vào ngày hôm qua, do dữ liệu người dùng được lưu ở trên nhiều hệ thống như web, application và cũng như trên các bộ phận khác nhau
  của doanh nghiệp. Khi đó chúng ta sẽ có nhiều nguồn dữ liệu khác nhau và cần một thuật toán để kết hợp các nguồn dữ liệu này để tổng hợp cho ra ước lượng
  số lượng cuối cùng.
  \begin{itemize}
    \item Bước 1: Tổng hợp dữ liệu
    Đầu tiên, chúng ta đã lưu trữ dữ liệu trên một dòng dữ liệu như thuật toán ở trên. 
    \item Bước 2: Tổng hợp HyperLogLog
    Tiếp theo, chúng tôi sẽ sẽ tiến hành tổng hợp các dữ liệu từ nhiều nơi khác nhau $\left< T_1, HLL_1\right>, \left< T_2, HLL_2\right>,...,\left< T_N, HLL_N\right>$, 
    trong đó $T_1, T_2,...,T_N$ là các khoảng thời gian giống nhau nên $T_1 = T_2 = T_N$ và đặt chung là $T$, và $HLL_1$ là dữ liệu HyperLogLog trong khoảng thời gian, ví dụ
    từ 12:00 ngày hôm qua cho đến 12:00 ngày hôm nay.  
    \item Bước 3: Ước lượng số phần tử
    Cuối cùng, sau khi đã tổng hợp dữ liệu thành một cấu trúc HyperLogLog mới ta tiến hành ước lượng số phần tử theo công thức:
    \[E = \alpha_m \cdot m^2 \cdot \left( \sum_{j=1}^{m} 2^{-M[j]} \right)^{-1}\]
    Trong đó: $E$ là ước lượng số lượng phần tử duy nhất, 
    $\alpha_m$ là hằng số, được chọn tùy thuộc vào số lượng bộ nhớ sử dụng.
    $m$ là số lượng register trong cấu trúc HyperLogLog, $M[j]$ là giá trị của register thứ j.
\end{itemize}
\end{frame}
\section{KẾ HOẠCH TRIỂN KHAI}
\begin{frame}{Kế hoạch triển khai}
% \begin{center}
\begin{tabular}{|c|c|l| }
    \multicolumn{3}{}{} \\ \hline
    \# & Tuần & Nội dung công việc \\ \hline
    1 & 1 - 2 &\vtop{\hbox{\strut Tìm hiểu thêm các bài báo liên quan mới nhất và bổ sung}\hbox{\strut cơ sở lý thuyết về các kỹ thuật ước lượng số lượng trên dòng dữ liệu}}\\
    \hline
    2 & 3 - 4 &\vtop{\hbox{\strut Thu thập dữ liệu, chuẩn hoá và tiền xử lý. Hiện thực bài toán 1 }\hbox{\strut ước lượng số lượng phần tử trên dòng dữ liệu}}\\
    \hline
    3 & 5 - 6 &\vtop{\hbox{\strut Hiện thực bài toán mở rộng để ước lượng số lượng phần tử trong một khoảng thời gian}\hbox{\strut  trên nhiều dòng dữ liệu. Đánh giá hiệu suất và độ chính xác.}}\\
    \hline
    4 & 7 - 8 &\vtop{\hbox{\strut Phân tích và so sánh kết quả, đánh giá ưu nhược điểm của từng kỹ thuật ước lượng.}\hbox{\strut Đề suất phương pháp hay tối ưu hiệu suất và độ chính xác.}}\\
    \hline
    5 & 9 - 10 &\vtop{\hbox{\strut Ứng dụng kết quả nghiên cứu.}\hbox{\strut Đề suất hướng phát triển và nghiên cứu tiếp theo.}}\\
    \hline
    6 & 11 - 12 &\vtop{\hbox{\strut Đề xuất và đánh giá các giải pháp}}\\
    \hline
    7 & 1 - 14 & Tổng hợp kết quả và viết báo cáo \\
    \hline
\end{tabular}
    \textit{Thời gian dự kiến làm luận văn là từ ngày \textbf{05/09/2024} đến \textbf{15/12/2024}}
% \end{center}
\end{frame}
\begin{frame}{Nội dung dự kiến của luận văn}
  \textbf{Chương 1: Giới thiệu}. Chương này nhấn mạnh về tầm quan trọng của việc phát triển kỹ thuật đếm số phần tử trên dòng dữ liệu trong ngữ cảnh 
dữ liệu lớn. Qua đó, ta sẽ hiểu rõ hơn về phạm vi và mục tiêu của nghiên cứu, cũng như cách tiếp cận vấn đề.\\

\textbf{Chương 2: Các công trình nghiên cứu liên quan.} Trong chương này, sẽ được trình bày các công trình nghiên cứu liên quan, khám phá các 
phương pháp giải quyết vấn đề cũng như những phạm vi và giới hạn của các nghiên cứu đó. Qua đó, ta có thể đánh giá tính khả thi của đề tài.\\

\textbf{Chương 3: Kiến thức nền tảng.} Trong chương này giới thiệu về dòng dữ liệu, và các tính chất của nó.
Các phương pháp truy vấn và xử lý dữ liệu trên dòng dữ liệu. Giới thiệu về HyperLogLog và nguyên lý hoạt động của nó. Các kiến thức cơ bản về ước lượng số lượng phần tử trong dữ liệu lớn,
và các khái niệm về đánh giá hiệu suất và độ chính xác trong đếm số phần tử trên dòng dữ liệu.\\

\textbf{Chương 4: Hiện thực và thử nghiệm. } Trong chương này sẽ trình bày chi tiết cách thức hiện thực của từng thuật toán.\\

\textbf{Chương 5: Kết quả và đánh giá.} Trong chương này sẽ nêu ra các kết quả đạt được của các kỹ thuật, cũng như phương pháp đánh giá 
dựa trên kết quả thực nghiệm.\\

\textbf{Chương 6: Kết luận.} Trong chương này sẽ tóm lại các ưu điểm và nhược điểm của mô hình và đưa ra các hướng nghiên cứu phát triển 
các kỹ thuật đếm số phần tử trong tương lai.
\end{frame}  

\section{Kết luận}
\begin{frame}{Kết luận}
  Việc giám sát và quản lý số lượng người dùng đóng vai trò quan trọng trong việc tối ưu hóa hiệu quả hoạt động, 
nâng cao trải nghiệm người dùng, hỗ trợ ra quyết định kinh doanh sáng suốt và đảm bảo an ninh mạng cho doanh nghiệp.\\

Phân bổ tài nguyên hợp lý: Đảm bảo hệ thống hoạt động ổn định, tránh quá tải, lãng phí tài nguyên, tối ưu hóa chi phí vận hành.\\

Nâng cao trải nghiệm người dùng: Giảm thiểu lỗi hệ thống, lag, giật, loading lâu, mang đến trải nghiệm mượt mà, thu hút và giữ chân khách hàng.\\

Phát hiện và khắc phục sự cố kịp thời: Nhận diện sớm các dấu hiệu bất thường, sự cố hệ thống, từ đó có biện pháp khắc phục nhanh chóng, hạn chế ảnh hưởng đến hoạt động kinh doanh.\\

Hiểu rõ hành vi người dùng: Phân tích dữ liệu truy cập, hành vi click chuột, sở thích, nhu cầu của người dùng để cá nhân hóa trải nghiệm, đề xuất sản phẩm/dịch vụ phù hợp, 
nâng cao hiệu quả marketing và dự báo xu hướng thị trường.\\

Hỗ trợ ra quyết định kinh doanh: Đánh giá hiệu quả chiến dịch marketing, điều chỉnh chiến lược phù hợp, dự báo nhu cầu thị trường, 
lập kế hoạch sản xuất, kinh doanh hiệu quả, tối ưu hóa nguồn lực, nâng cao lợi nhuận cho doanh nghiệp.\\

Sự bùng nổ của kỷ nguyên dữ liệu lớn đặt ra nhu cầu cấp thiết cho các kỹ thuật xử lý dữ liệu hiệu quả, chính xác. 
Đề tài "NGHIÊN CỨU PHÁT TRIỂN KỸ THUẬT ĐẾM SỐ PHẦN TỬ TRÊN DÒNG DỮ LIỆU" là một đóng góp quan trọng cho lĩnh vực khoa học máy tính. 
Các thuật toán mới được phát triển trong đề tài này giúp nâng cao hiệu quả và độ chính xác của việc xử lý dữ liệu, mang đến những đóng góp 
to lớn cho lĩnh vực khoa học máy tính và mở ra cánh cửa cho vô số ứng dụng thực tiễn.
\end{frame}

\begin{frame}[standout]
  Questions?
\end{frame}

\appendix

\begin{frame}[fragile]{Backup slides}
  Sometimes, it is useful to add slides at the end of your presentation to
  refer to during audience questions.

  The best way to do this is to include the \verb|appendixnumberbeamer|
  package in your preamble and call \verb|\appendix| before your backup slides.

  \themename will automatically turn off slide numbering and progress bars for
  slides in the appendix.
\end{frame}

\begin{frame}[allowframebreaks]{Tài liệu tham khảo}
  \bibliography{ref}
  \bibliographystyle{abbrv}
\end{frame}

\end{document}
