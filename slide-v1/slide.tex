\documentclass[10pt]{beamer}

\usetheme{metropolis}
\usepackage{appendixnumberbeamer}
\usepackage[utf8]{vietnam}
\usepackage{booktabs}
\usepackage[scale=2]{ccicons}

\usepackage{pgfplots}
\usepgfplotslibrary{dateplot}

\usepackage{xspace}
\newcommand{\themename}{\textbf{\textsc{metropolis}}\xspace}

\title{Đề cương luận văn thạc sĩ}
\subtitle{Nghiên cứu phát triển kỹ thuật đếm số phần tử \\ \hspace{2.4cm} trên dòng dữ liệu}
\date{Người hướng dẫn khoa học: \hspace{0.5cm}PGS. TS. THOẠI NAM}
\author{Học viên: Lê Anh Quốc \hspace{1cm} ID: 2070428}
%\institute{\today}
\titlegraphic{\hfill\includegraphics[height=1.5cm]{hcmut.png}}

\begin{document}

\maketitle

\begin{frame}{Outline}
  \setbeamertemplate{section in toc}[sections numbered]
  \tableofcontents[hideallsubsections]
\end{frame}

\section{Giới thiệu}

\begin{frame}[fragile]{Giới thiệu}
Ngày nay, các ứng dụng và dịch vụ trực tuyến đóng vai trò ngày càng quan trọng trong cuộc sống của
con người. Chúng ta sử dụng mạng xã hội để kết nối với bạn bè và chia sẻ thông tin, mua sắm trực tuyến
để tiết kiệm thời gian và tiền bạc, hay xem phim và chơi game trực tuyến để giải trí. Để đánh giá hiệu quả
hoạt động của các ứng dụng và dịch vụ này, một trọng những chỉ số quan trọng nhất là số lượng người dùng
hoạt động.
Việc theo dõi số lượng người dùng hoạt động trong một khoảng thời gian nhất định trên một dòng dữ
liệu (data stream) là một yêu cầu quan trọng đối với nhiều ứng dụng và dịch vụ trực tuyến, hiệu quả của
các chiến dịch marketing, và hỗ trợ ra quyết định kinh doanh.
Ví dụ, trong các ứng dụng mạng xã hội, số lượng người dùng hoạt động cho thấy mức độ tương tác và sự
quan tâm của người dùng đối với nền tảng. Trong các dịch vụ thương mại điện tử, số lượng người dùng hoạt
động cho thấy hiệu quả và các chiến dịch quảng cáo và khuyến mãi.
Tuy nhiên, việc đếm số lượng người dùng không phải là một nhiệm vụ đơn giản, đặc biệt là khi dữ liệu lớn
và tốc độ truy cập cao. Các phương pháp truyền thống như lưu trữ và truy vấn trực tiếp vào cơ sở dữ liệu
có thể gặp nhiều hạn chế về hiệu suất và khả năng mở rộng.
\end{frame}
\begin{frame}[fragile]{Giới thiệu}
  Trong nhiều trường hợp, cần phải tổng hợp số lượng người dùng trên nhiều dòng dữ liệu khác nhau. Việc
này giúp có được bức tranh toàn cảnh về hoạt động của người dùng trên toàn hệ thống, từ đó đưa ra các
phân tích và đánh giá chính xác hơn.
Ví dụ, trong hệ thống thương mại điện tử, cần tổng hợp số lượng người dùng từ các trang web, ứng dụng di
động và API khác nhau để có được số lượng người dùng hoạt động thực tế trên toàn hệ thống. Tuy nhiên,
việc tổng hợp dữ liệu từ nhiều nguồn khác nhau có thể gặp thách thức về đồng bộ hóa dữ liệu, xử lý dữ liệu
bị thiếu hoặc lỗi, và đảm bảo tính nhất quán của kết quả.
Ngoài ra, có thể cần phải đếm số lượng người dùng trên nhiều khoảng thời gian khác nhau trên một hoặc
nhiều dòng dữ liệu khác nhau. Việc này giúp phân tích chi tiết hơn hoạt động của người dùng theo thời gian,
theo khu vực hoặc theo tiêu chí khác.
\end{frame}

\begin{frame}[fragile]{Giới thiệu}
Ví dụ, trong một ứng dụng phát trực tiếp, cần đếm số lượng người dùng hoạt động theo giờ hoặc từng phân
đoạn chương trình để đánh giá mức độ quan tâm của người xem. Tuy nhiên, việc phân chia và xử lý dữ liệu
theo nhiều đoạn có thể làm tăng độ phức tạp của thuật toán và ảnh hưởng đến hiệu suất của hệ thống. Do
đó, cần phải có một giải pháp đếm số lượng phần tử trên dòng dữ liệu đạt hiệu suất cao và tin cậy, từ đó
có thể ứng dụng rộng rãi trong các hệ thống khác nhau như mạng xã hội, thương mại điện tử, chương trình
phát trực tiếp, hệ thống giám sát và hệ thống giao thông thông minh.
\end{frame}

\begin{frame}[fragile]{Sections}
  Sections group slides of the same topic

  \begin{verbatim}    \section{Elements}\end{verbatim}

  for which \themename provides a nice progress indicator \ldots
\end{frame}
\section{Các công trình nghiên cứu liên quan}
\begin{frame}[fragile]{LogLog \cite{durand2003loglog}}
  Thuật toán LogLog cho phép ước lượng số lượng từ vựng khác nhau trong toàn bộ tác phẩm
  của Shakespeare chỉ trong một lần quét và với độ chính xác cỡ vài phần trăm, sử dụng một lượng bộ nhớ
  phụ nhỏ. Phiên bản cơ bản đã được xác minh qua phân tích toàn diện và có phiên bản tối ưu hóa có khả
  năng song song.
\end{frame}
\begin{frame}[fragile]{HyperLogLog \cite{flajolet2007hyperloglog}}
  Thuật toán HYPERLOGLOG là một thuật toán xác suất gần tối ưu, được thiết kế để ước lượng số lượng các phần tử khác nhau trong các tập dữ liệu rất lớn. 
  Sử dụng bộ nhớ phụ có kích thước m đơn vị, HYPERLOGLOG thực hiện một lần quét qua dữ liệu và tạo ra một ước lượng về số lượng phần tử khác nhau với độ chính xác tương đối là khoảng $\frac{1.04}{\sqrt{m}}$. 
  Thuật toán này có khả năng ước lượng số lượng phần tử lớn hơn $10^9$ với độ chính xác khoảng 2\% chỉ sử dụng 1.5 kilobytes bộ nhớ, đồng thời có khả năng song song hoá tối ưu 
  và thích nghi với mô hình cửa sổ trượt (sliding windown).
\end{frame}
\begin{frame}[fragile]{HyperLogLog++ \cite{heule2013hyperloglog}}
  Bài báo giới thiệu một thuật toán mới ước lượng số lượng luồng hoạt động trong dòng dữ liệu, sử dụng cơ chế cửa sổ trượt kết hợp với 
  thuật toán HyperLogLog. Thuật toán này có độ chính xác cao, lỗi tiêu chuẩn khoảng $\frac{1.04}{\sqrt{m}}$, với m là số lượng thanh ghi trong bộ nhớ. Dù cần bộ nhớ bổ sung 
  so với HyperLogLog, tổng bộ nhớ cần thiết không vượt quá $5m\ln(\frac{n}{m})$ byte, với n là số luồng thực sự trong cửa sổ trượt. Kết quả lý thuyết được xác minh trên cả dữ liệu thực 
  và tổng hợp.
\end{frame}
\begin{frame}[fragile]{Sliding HyperLogLog \cite{chabchoub2010sliding}}
  Bài báo giới thiệu một thuật toán mới ước lượng số lượng luồng hoạt động trong dòng dữ liệu, sử dụng cơ chế cửa sổ trượt kết hợp với 
  thuật toán HyperLogLog. Thuật toán này có độ chính xác cao, lỗi tiêu chuẩn khoảng $\frac{1.04}{\sqrt{m}}$, với m là số lượng thanh ghi trong bộ nhớ. Dù cần bộ nhớ bổ sung 
  so với HyperLogLog, tổng bộ nhớ cần thiết không vượt quá $5m\ln(\frac{n}{m})$ byte, với n là số luồng thực sự trong cửa sổ trượt. Kết quả lý thuyết được xác minh trên cả dữ liệu thực 
  và tổng hợp.
\end{frame}
\begin{frame}[fragile]{ExaLogLog \cite{ertl2024exaloglog}}
  ExaLogLog là một cấu trúc dữ liệu mới cho việc đếm độc lập xấp xỉ, tương tự như HyperLogLog, nhưng tiêu tốn ít hơn 43\% không gian với cùng lỗi ước lượng.
\end{frame}
\section{Phát biểu bài toán}
\begin{frame}[fragile]{Phát biểu bài toán}
\textbf{Bài toán 1:} Phát triển thuật toán để ước lượng số lượng phần tử (cardinality estimation) trong một khoảng thời gian trên một dòng dữ liệu (data stream).\\
\textbf{Bài toán 2:} Mở rộng thuật toán để ước lượng số lượng phần tử trong một khoảng thời gian trên nhiều dòng dữ liệu.
\end{frame}
\section{Mục tiêu, đối tượng và giới hạn nghiên cứu}

\begin{frame}{Mục tiêu, đối tượng và giới hạn nghiên cứu}
	\themename supports 4 different title formats:
	\begin{itemize}
		\item Regular
		\item \textsc{Small caps}
		\item \textsc{all small caps}
		\item ALL CAPS
	\end{itemize}
	They can either be set at once for every title type or individually.
\end{frame}

{
    \metroset{titleformat frame=smallcaps}
\begin{frame}{Small caps}
	This frame uses the \texttt{smallcaps} title format.

	\begin{alertblock}{Potential Problems}
		Be aware that not every font supports small caps. If for example you typeset your presentation with pdfTeX and the Computer Modern Sans Serif font, every text in small caps will be typeset with the Computer Modern Serif font instead.
	\end{alertblock}
\end{frame}
}

{
\metroset{titleformat frame=allsmallcaps}
\begin{frame}{All small caps}
	This frame uses the \texttt{allsmallcaps} title format.

	\begin{alertblock}{Potential problems}
		As this title format also uses small caps you face the same problems as with the \texttt{smallcaps} title format. Additionally this format can cause some other problems. Please refer to the documentation if you consider using it.

		As a rule of thumb: just use it for plaintext-only titles.
	\end{alertblock}
\end{frame}
}

{
\metroset{titleformat frame=allcaps}
\begin{frame}{All caps}
	This frame uses the \texttt{allcaps} title format.

	\begin{alertblock}{Potential Problems}
		This title format is not as problematic as the \texttt{allsmallcaps} format, but basically suffers from the same deficiencies. So please have a look at the documentation if you want to use it.
	\end{alertblock}
\end{frame}
}

\section{Phát biểu bài toán}

\begin{frame}[fragile]{Phát biểu bài toán}
      \begin{verbatim}The theme provides sensible defaults to
\emph{emphasize} text, \alert{accent} parts
or show \textbf{bold} results.\end{verbatim}

  \begin{center}becomes\end{center}

  The theme provides sensible defaults to \emph{emphasize} text,
  \alert{accent} parts or show \textbf{bold} results.
\end{frame}

\begin{frame}{Font feature test}
  \begin{itemize}
    \item Regular
    \item \textit{Italic}
    \item \textsc{Small Caps}
    \item \textbf{Bold}
    \item \textbf{\textit{Bold Italic}}
    \item \textbf{\textsc{Bold Small Caps}}
    \item \texttt{Monospace}
    \item \texttt{\textit{Monospace Italic}}
    \item \texttt{\textbf{Monospace Bold}}
    \item \texttt{\textbf{\textit{Monospace Bold Italic}}}
  \end{itemize}
\end{frame}

\begin{frame}{Lists}
  \begin{columns}[T,onlytextwidth]
    \column{0.33\textwidth}
      Items
      \begin{itemize}
        \item Milk \item Eggs \item Potatoes
      \end{itemize}

    \column{0.33\textwidth}
      Enumerations
      \begin{enumerate}
        \item First, \item Second and \item Last.
      \end{enumerate}

    \column{0.33\textwidth}
      Descriptions
      \begin{description}
        \item[PowerPoint] Meeh. \item[Beamer] Yeeeha.
      \end{description}
  \end{columns}
\end{frame}
\begin{frame}{Animation}
  \begin{itemize}[<+- | alert@+>]
    \item \alert<4>{This is\only<4>{ really} important}
    \item Now this
    \item And now this
  \end{itemize}
\end{frame}
\begin{frame}{Figures}
  \begin{figure}
    \newcounter{density}
    \setcounter{density}{20}
    \begin{tikzpicture}
      \def\couleur{alerted text.fg}
      \path[coordinate] (0,0)  coordinate(A)
                  ++( 90:5cm) coordinate(B)
                  ++(0:5cm) coordinate(C)
                  ++(-90:5cm) coordinate(D);
      \draw[fill=\couleur!\thedensity] (A) -- (B) -- (C) --(D) -- cycle;
      \foreach \x in {1,...,40}{%
          \pgfmathsetcounter{density}{\thedensity+20}
          \setcounter{density}{\thedensity}
          \path[coordinate] coordinate(X) at (A){};
          \path[coordinate] (A) -- (B) coordinate[pos=.10](A)
                              -- (C) coordinate[pos=.10](B)
                              -- (D) coordinate[pos=.10](C)
                              -- (X) coordinate[pos=.10](D);
          \draw[fill=\couleur!\thedensity] (A)--(B)--(C)-- (D) -- cycle;
      }
    \end{tikzpicture}
    \caption{Rotated square from
    \href{http://www.texample.net/tikz/examples/rotated-polygons/}{texample.net}.}
  \end{figure}
\end{frame}
\begin{frame}{Tables}
  \begin{table}
    \caption{Largest cities in the world (source: Wikipedia)}
    \begin{tabular}{@{} lr @{}}
      \toprule
      City & Population\\
      \midrule
      Mexico City & 20,116,842\\
      Shanghai & 19,210,000\\
      Peking & 15,796,450\\
      Istanbul & 14,160,467\\
      \bottomrule
    \end{tabular}
  \end{table}
\end{frame}
\begin{frame}{Blocks}
  Three different block environments are pre-defined and may be styled with an
  optional background color.

  \begin{columns}[T,onlytextwidth]
    \column{0.5\textwidth}
      \begin{block}{Default}
        Block content.
      \end{block}

      \begin{alertblock}{Alert}
        Block content.
      \end{alertblock}

      \begin{exampleblock}{Example}
        Block content.
      \end{exampleblock}

    \column{0.5\textwidth}

      \metroset{block=fill}

      \begin{block}{Default}
        Block content.
      \end{block}

      \begin{alertblock}{Alert}
        Block content.
      \end{alertblock}

      \begin{exampleblock}{Example}
        Block content.
      \end{exampleblock}

  \end{columns}
\end{frame}
\begin{frame}{Math}
  \begin{equation*}
    e = \lim_{n\to \infty} \left(1 + \frac{1}{n}\right)^n
  \end{equation*}
\end{frame}
\begin{frame}{Line plots}
  \begin{figure}
    \begin{tikzpicture}
      \begin{axis}[
        mlineplot,
        width=0.9\textwidth,
        height=6cm,
      ]

        \addplot {sin(deg(x))};
        \addplot+[samples=100] {sin(deg(2*x))};

      \end{axis}
    \end{tikzpicture}
  \end{figure}
\end{frame}
\begin{frame}{Bar charts}
  \begin{figure}
    \begin{tikzpicture}
      \begin{axis}[
        mbarplot,
        xlabel={Foo},
        ylabel={Bar},
        width=0.9\textwidth,
        height=6cm,
      ]

      \addplot plot coordinates {(1, 20) (2, 25) (3, 22.4) (4, 12.4)};
      \addplot plot coordinates {(1, 18) (2, 24) (3, 23.5) (4, 13.2)};
      \addplot plot coordinates {(1, 10) (2, 19) (3, 25) (4, 15.2)};

      \legend{lorem, ipsum, dolor}

      \end{axis}
    \end{tikzpicture}
  \end{figure}
\end{frame}
\begin{frame}{Quotes}
  \begin{quote}
    Veni, Vidi, Vici
  \end{quote}
\end{frame}

{%
\setbeamertemplate{frame footer}{My custom footer}
\begin{frame}[fragile]{Frame footer}
    \themename defines a custom beamer template to add a text to the footer. It can be set via
    \begin{verbatim}\setbeamertemplate{frame footer}{My custom footer}\end{verbatim}
\end{frame}
}

\begin{frame}{References}
  Some references to showcase [allowframebreaks] \cite{durand2003loglog, flajolet2007hyperloglog, chabchoub2010sliding, ertl2024exaloglog, heule2013hyperloglog}
\end{frame}

\section{Kiến thức nền tảng}

\begin{frame}{Kiến thức nền tảng}

  Get the source of this theme and the demo presentation from

  \begin{center}\url{github.com/matze/mtheme}\end{center}

  The theme \emph{itself} is licensed under a
  \href{http://creativecommons.org/licenses/by-sa/4.0/}{Creative Commons
  Attribution-ShareAlike 4.0 International License}.

  \begin{center}\ccbysa\end{center}

\end{frame}

\section{Phương pháp thực hiện}

\begin{frame}{Phương pháp thực hiện}

  Get the source of this theme and the demo presentation from

  \begin{center}\url{github.com/matze/mtheme}\end{center}

  The theme \emph{itself} is licensed under a
  \href{http://creativecommons.org/licenses/by-sa/4.0/}{Creative Commons
  Attribution-ShareAlike 4.0 International License}.

  \begin{center}\ccbysa\end{center}

\end{frame}

\begin{frame}[standout]
  Questions?
\end{frame}

\appendix

\begin{frame}[fragile]{Backup slides}
  Sometimes, it is useful to add slides at the end of your presentation to
  refer to during audience questions.

  The best way to do this is to include the \verb|appendixnumberbeamer|
  package in your preamble and call \verb|\appendix| before your backup slides.

  \themename will automatically turn off slide numbering and progress bars for
  slides in the appendix.
\end{frame}

\begin{frame}[allowframebreaks]{Tài liệu tham khảo}
  \bibliography{ref}
  \bibliographystyle{abbrv}
\end{frame}

\end{document}
