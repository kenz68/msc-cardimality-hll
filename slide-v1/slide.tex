\documentclass[10pt]{beamer}
\usepackage[linesnumbered,algoruled,boxed,lined]{algorithm2e}
\usetheme{metropolis}
\usepackage{appendixnumberbeamer}
\usepackage[utf8]{vietnam}
\usepackage{booktabs}
\usepackage[scale=2]{ccicons}

\usepackage{pgfplots}
\usepgfplotslibrary{dateplot}

\usepackage{xspace}
\newcommand{\themename}{\textbf{\textsc{metropolis}}\xspace}

\title{Đề cương luận văn thạc sĩ}
\subtitle{Nghiên cứu phát triển kỹ thuật đếm số phần tử \\ \hspace{2.4cm} trên dòng dữ liệu}
\date{Người hướng dẫn khoa học: \hspace{0.5cm}PGS. TS. THOẠI NAM}
\author{Học viên: Lê Anh Quốc \hspace{1cm} ID: 2070428}
%\institute{\today}
\titlegraphic{\hfill\includegraphics[height=1.5cm]{hcmut.png}}

\begin{document}

\maketitle

\begin{frame}{Outline}
  \setbeamertemplate{section in toc}[sections numbered]
  \tableofcontents[hideallsubsections]
\end{frame}

\section{Giới thiệu}

\begin{frame}[fragile]{Giới thiệu}
Ngày nay, các ứng dụng và dịch vụ trực tuyến đóng vai trò ngày càng quan trọng trong cuộc sống của
con người. Chúng ta sử dụng mạng xã hội để kết nối với bạn bè và chia sẻ thông tin, mua sắm trực tuyến
để tiết kiệm thời gian và tiền bạc, hay xem phim và chơi game trực tuyến để giải trí. Để đánh giá hiệu quả
hoạt động của các ứng dụng và dịch vụ này, một trọng những chỉ số quan trọng nhất là số lượng người dùng
hoạt động.
Việc theo dõi số lượng người dùng hoạt động trong một khoảng thời gian nhất định trên một dòng dữ
liệu (data stream) là một yêu cầu quan trọng đối với nhiều ứng dụng và dịch vụ trực tuyến, hiệu quả của
các chiến dịch marketing, và hỗ trợ ra quyết định kinh doanh.
Ví dụ, trong các ứng dụng mạng xã hội, số lượng người dùng hoạt động cho thấy mức độ tương tác và sự
quan tâm của người dùng đối với nền tảng. Trong các dịch vụ thương mại điện tử, số lượng người dùng hoạt
động cho thấy hiệu quả và các chiến dịch quảng cáo và khuyến mãi.
Tuy nhiên, việc đếm số lượng người dùng không phải là một nhiệm vụ đơn giản, đặc biệt là khi dữ liệu lớn
và tốc độ truy cập cao. Các phương pháp truyền thống như lưu trữ và truy vấn trực tiếp vào cơ sở dữ liệu
có thể gặp nhiều hạn chế về hiệu suất và khả năng mở rộng.
\end{frame}
\begin{frame}[fragile]{Giới thiệu}
  Trong nhiều trường hợp, cần phải tổng hợp số lượng người dùng trên nhiều dòng dữ liệu khác nhau. Việc
này giúp có được bức tranh toàn cảnh về hoạt động của người dùng trên toàn hệ thống, từ đó đưa ra các
phân tích và đánh giá chính xác hơn.
Ví dụ, trong hệ thống thương mại điện tử, cần tổng hợp số lượng người dùng từ các trang web, ứng dụng di
động và API khác nhau để có được số lượng người dùng hoạt động thực tế trên toàn hệ thống. Tuy nhiên,
việc tổng hợp dữ liệu từ nhiều nguồn khác nhau có thể gặp thách thức về đồng bộ hóa dữ liệu, xử lý dữ liệu
bị thiếu hoặc lỗi, và đảm bảo tính nhất quán của kết quả.
Ngoài ra, có thể cần phải đếm số lượng người dùng trên nhiều khoảng thời gian khác nhau trên một hoặc
nhiều dòng dữ liệu khác nhau. Việc này giúp phân tích chi tiết hơn hoạt động của người dùng theo thời gian,
theo khu vực hoặc theo tiêu chí khác.
\end{frame}

\begin{frame}[fragile]{Giới thiệu}
Ví dụ, trong một ứng dụng phát trực tiếp, cần đếm số lượng người dùng hoạt động theo giờ hoặc từng phân
đoạn chương trình để đánh giá mức độ quan tâm của người xem. Tuy nhiên, việc phân chia và xử lý dữ liệu
theo nhiều đoạn có thể làm tăng độ phức tạp của thuật toán và ảnh hưởng đến hiệu suất của hệ thống. Do
đó, cần phải có một giải pháp đếm số lượng phần tử trên dòng dữ liệu đạt hiệu suất cao và tin cậy, từ đó
có thể ứng dụng rộng rãi trong các hệ thống khác nhau như mạng xã hội, thương mại điện tử, chương trình
phát trực tiếp, hệ thống giám sát và hệ thống giao thông thông minh.
\end{frame}
\section{Các công trình nghiên cứu liên quan}
\begin{frame}[fragile]{LogLog \cite{durand2003loglog}}
  Thuật toán LogLog cho phép ước lượng số lượng từ vựng khác nhau trong toàn bộ tác phẩm
  của Shakespeare chỉ trong một lần quét và với độ chính xác cỡ vài phần trăm, sử dụng một lượng bộ nhớ
  phụ nhỏ. Phiên bản cơ bản đã được xác minh qua phân tích toàn diện và có phiên bản tối ưu hóa có khả
  năng song song.
\end{frame}
\begin{frame}[fragile]{HyperLogLog \cite{flajolet2007hyperloglog}}
  Thuật toán HYPERLOGLOG là một thuật toán xác suất gần tối ưu, được thiết kế để ước lượng số lượng các phần tử khác nhau trong các tập dữ liệu rất lớn. 
  Sử dụng bộ nhớ phụ có kích thước m đơn vị, HYPERLOGLOG thực hiện một lần quét qua dữ liệu và tạo ra một ước lượng về số lượng phần tử khác nhau với độ chính xác tương đối là khoảng $\frac{1.04}{\sqrt{m}}$. 
  Thuật toán này có khả năng ước lượng số lượng phần tử lớn hơn $10^9$ với độ chính xác khoảng 2\% chỉ sử dụng 1.5 kilobytes bộ nhớ, đồng thời có khả năng song song hoá tối ưu 
  và thích nghi với mô hình cửa sổ trượt (sliding windown).
\end{frame}
\begin{frame}[fragile]{HyperLogLog++ \cite{heule2013hyperloglog}}
  Bài báo giới thiệu một thuật toán mới ước lượng số lượng luồng hoạt động trong dòng dữ liệu, sử dụng cơ chế cửa sổ trượt kết hợp với 
  thuật toán HyperLogLog. Thuật toán này có độ chính xác cao, lỗi tiêu chuẩn khoảng $\frac{1.04}{\sqrt{m}}$, với m là số lượng thanh ghi trong bộ nhớ. Dù cần bộ nhớ bổ sung 
  so với HyperLogLog, tổng bộ nhớ cần thiết không vượt quá $5m\ln(\frac{n}{m})$ byte, với n là số luồng thực sự trong cửa sổ trượt. Kết quả lý thuyết được xác minh trên cả dữ liệu thực 
  và tổng hợp.
\end{frame}
\begin{frame}[fragile]{Sliding HyperLogLog \cite{chabchoub2010sliding}}
  Bài báo giới thiệu một thuật toán mới ước lượng số lượng luồng hoạt động trong dòng dữ liệu, sử dụng cơ chế cửa sổ trượt kết hợp với 
  thuật toán HyperLogLog. Thuật toán này có độ chính xác cao, lỗi tiêu chuẩn khoảng $\frac{1.04}{\sqrt{m}}$, với m là số lượng thanh ghi trong bộ nhớ. Dù cần bộ nhớ bổ sung 
  so với HyperLogLog, tổng bộ nhớ cần thiết không vượt quá $5m\ln(\frac{n}{m})$ byte, với n là số luồng thực sự trong cửa sổ trượt. Kết quả lý thuyết được xác minh trên cả dữ liệu thực 
  và tổng hợp.
\end{frame}
\begin{frame}[fragile]{ExaLogLog \cite{ertl2024exaloglog}}
  ExaLogLog là một cấu trúc dữ liệu mới cho việc đếm độc lập xấp xỉ, tương tự như HyperLogLog, nhưng tiêu tốn ít hơn 43\% không gian với cùng lỗi ước lượng.
\end{frame}
\section{Phát biểu bài toán}
\begin{frame}[fragile]{Phát biểu bài toán}
\textbf{Bài toán 1:} Phát triển thuật toán để ước lượng số lượng phần tử (cardinality estimation) trong một khoảng thời gian trên một dòng dữ liệu (data stream).\\
\textbf{Bài toán 2:} Mở rộng thuật toán để ước lượng số lượng phần tử trong một khoảng thời gian trên nhiều dòng dữ liệu.
\end{frame}
\section{Mục tiêu, đối tượng và giới hạn nghiên cứu}

\begin{frame}{Mục tiêu, đối tượng và giới hạn nghiên cứu}
	\themename supports 4 different title formats:
	\begin{itemize}
		\item Regular
		\item \textsc{Small caps}
		\item \textsc{all small caps}
		\item ALL CAPS
	\end{itemize}
	They can either be set at once for every title type or individually.
\end{frame}

{
    \metroset{titleformat frame=smallcaps}
\begin{frame}{Small caps}
	This frame uses the \texttt{smallcaps} title format.

	\begin{alertblock}{Potential Problems}
		Be aware that not every font supports small caps. If for example you typeset your presentation with pdfTeX and the Computer Modern Sans Serif font, every text in small caps will be typeset with the Computer Modern Serif font instead.
	\end{alertblock}
\end{frame}
}

{
\metroset{titleformat frame=allsmallcaps}
\begin{frame}{All small caps}
	This frame uses the \texttt{allsmallcaps} title format.

	\begin{alertblock}{Potential problems}
		As this title format also uses small caps you face the same problems as with the \texttt{smallcaps} title format. Additionally this format can cause some other problems. Please refer to the documentation if you consider using it.

		As a rule of thumb: just use it for plaintext-only titles.
	\end{alertblock}
\end{frame}
}

{
\metroset{titleformat frame=allcaps}
\begin{frame}{All caps}
	This frame uses the \texttt{allcaps} title format.

	\begin{alertblock}{Potential Problems}
		This title format is not as problematic as the \texttt{allsmallcaps} format, but basically suffers from the same deficiencies. So please have a look at the documentation if you want to use it.
	\end{alertblock}
\end{frame}
}

\section{Cơ sở lý thuyết}

\begin{frame}[fragile]{Cơ sở lý thuyết}
Các thuật toán xác suất phổ biến nhất để ước lượng số lượng được sử dụng trong 
thực tế là họ các thuật toán LogLog bao gồm thuật toán \textit{LogLog}, 
được đề xuất bởi Marianne Durand và Philippe Flajolet vào năm 2003 \cite{durand2003loglog}, 
và các kế thừa của nó \textit{HyperLogLog} và \textit{HyperLogLog++}.\\
Các thuật toán này sử dụng một phương pháp tương tự như thuật toán Đếm Xác Suất trong việc ước lượng số lượng $n$ bằng cách quan sát số lượng lớn nhất 
của các số không dấu hàng đầu trong biểu diễn nhị phân của các giá trị. Tất cả chúng đều yêu cầu một bộ nhớ phụ trợ và thực hiện một lần duyệt qua dữ liệu 
để tạo ra một ước lượng về số lượng.\\

Mỗi phần tử trong tập dữ liệu được tiền xử lý bằng cách áp dụng một hàm băm $h$ chuyển đổi các phần tử thành số nguyên phân bố đều đặn đủ 
trên một phạm vi scala $\{0,1,...,2^M-1\}$ hoặc, tương đương, trên tập hợp các chuỗi nhị phân có độ dài M:
\[
    h(x) = j = \sum\limits_{k=0}^{M-1}j_k\cdot2^k := \left(i_0i_1...i_{M-1}\right)_2,i_k \in \{0,1\}
\]
\end{frame}
\begin{frame}[fragile]{Cơ sở lý thuyết}
Đầu tiên, chúng ta chia tập dữ liệu ban đầu hoặc dòng dữ liệu đầu vào thành một số 
tập con, mỗi tập con này được lập chỉ mục bởi một trong $m$ bộ đếm đơn giản. 
Sau đó, theo phương pháp trung bình ngẫu nhiên, vì có một hàm băm đơn giản, chúng ta 
chọn bộ đếm cho phần tử cụ thể $x$ bằng cách sử dụng một phần của giá trị băm 
của nó $h(x)$, trong khi phần còn lại được sử dụng để cập nhật bộ đếm tương ứng.\\
Tất cả các thuật toán được thảo luận ở đây dựa trên việc quan sát các mẫu $0^k1$ 
xuất hiện ở đầu của các giá trị cho bộ đếm cụ thể, và gán mỗi mẫu với chỉ số của nó, 
gọi là hạng (rank). Hạng tương đương với vị trí bit 1 đầu tiên từ trái qua phải 
trong biểu diễn nhị phân của giá trị băm của phần tử được lập chỉ mục và có thể được 
tính bằng công thức:
\[
    rank(i)=\left\{
                \begin{array}{ll}
                    \min\limits_{i_k\neq 0}, \indent\text{for } i > 0,\\
                    M\indent\indent\text{for } i = 0
                \end{array}
            \right.
\]
Mỗi bộ đếm đơn giản xây dựng quan sát về số lượng của riêng mình dựa trên các hạng 
đã nhìn thấy, ước lượng cuối cùng của số lượng được tạo ra từ quan sát bằng một 
hàm đánh giá.Liên quan đến lưu trữ, thuật toán \textit{LogLog} đề xuất một 
giải pháp tiết kiệm lưu trữ cùng với một hàm đánh giá và phương pháp hiệu chỉnh 
sai lệch tốt hơn.
\end{frame}

\begin{frame}{Font feature test}
\end{frame}
\section{Phương pháp thực hiện}
\begin{frame}[fragile]{Bài toán 1}
Phát triển thuật toán để ước lượng số lượng phần tử (cardinality estimation) trong 
một khoảng thời gian trên một dòng dữ liệu (data stream): Trong phương pháp này, 
chúng tôi sẽ trình bày quá trình phát triển thuật toán sử dụng HyperLogLog để 
ước lượng số lượng phần tử trên một dòng dữ liệu. Bằng cách sử dụng cấu trúc dữ liệu 
HyperLogLog thep khung thời gian, ví dụ như mỗi giờ hoặc mỗi phút và các kỹ thuật 
tối ưu, chúng tôi xây dựng một thuật toán hiệu quả và chính xác để đếm số lượng 
phần tử duy nhất trong dữ liệu dòng.
\begin{itemize}
  \item Bước 1: Xác định khoảng thời gian
  Đầu tiên, chúng tôi sẽ xác định khoảng thời gian mà chúng tôi muốn đếm số lượng 
  phần tử. Ví dụ, mỗi giờ hoặc mỗi phút.
  \item Bước 2: Lưu trữ HyperLogLog
  Tiếp theo, chúng tôi sẽ lưu trữ cấu trúc HyperLogLog cho mỗi khoảng thời gian. 
  Cấu trúc dữ liệu sẽ bao gồm cặp $\left< T_1, HLL_1\right>$, trong đó $T_1$ là 
  thời điểm đại diện cho khung thời gian cụ thể.
  \item Bước 3: Sử dụng kết quả
  Cuối cùng, khi cần, chúng tôi có thể truy vấn và sử dụng kết quả từ các cấu trúc 
  HyperLogLog lưu trữ theo khung thời gian để ước lượng số lượng phần tử trong 
  mỗi khoảng thời gian.
\end{itemize}
\end{frame}
\begin{frame}[fragile]{Bài toán 2}
  Mở rộng thuật toán để ước lượng số lượng phần tử trong một khoảng thời gian trên nhiều dòng dữ liệu:
  Trong phương pháp này, chúng tôi sẽ sẽ mở rộng thuật toán 1 để ước lượng trên nhiều dòng dữ liệu. Ví dụ khi chúng ta cần biết có bao nhiêu người dùng
  đã đăng nhập vào hệ thống vào ngày hôm qua, do dữ liệu người dùng được lưu ở trên nhiều hệ thống như web, application và cũng như trên các bộ phận khác nhau
  của doanh nghiệp. Khi đó chúng ta sẽ có nhiều nguồn dữ liệu khác nhau và cần một thuật toán để kết hợp các nguồn dữ liệu này để tổng hợp cho ra ước lượng
  số lượng cuối cùng.
  \begin{itemize}
      \item Bước 1: Tổng hợp dữ liệu
      Đầu tiên, chúng ta đã lưu trữ dữ liệu trên một dòng dữ liệu như thuật toán ở trên. 
      \item Bước 2: Tổng hợp HyperLogLog
      Tiếp theo, chúng tôi sẽ sẽ tiến hành tổng hợp các dữ liệu từ nhiều nơi khác nhau $\left< T_1, HLL_1\right>, \left< T_2, HLL_2\right>,...,\left< T_N, HLL_N\right>$, 
      trong đó $T_1, T_2,...,T_N$ là các khoảng thời gian giống nhau nên $T_1 = T_2 = T_N$ và đặt chung là $T$, và $HLL_1$ là dữ liệu HyperLogLog trong khoảng thời gian, ví dụ
      từ 12:00 ngày hôm qua cho đến 12:00 ngày hôm nay.  
      \item Bước 3: Ước lượng số phần tử
      Cuối cùng, sau khi đã tổng hợp dữ liệu thành một cấu trúc HyperLogLog mới ta tiến hành ước lượng số phần tử theo công thức:
      \[E = \alpha_m \cdot m^2 \cdot \left( \sum_{j=1}^{m} 2^{-M[j]} \right)^{-1}\]
      Trong đó: $E$ là ước lượng số lượng phần tử duy nhất, 
      $\alpha_m$ là hằng số, được chọn tùy thuộc vào số lượng bộ nhớ sử dụng.
      $m$ là số lượng register trong cấu trúc HyperLogLog, $M[j]$ là giá trị của register thứ j.
\end{itemize}
\end{frame}
\section{Kế hoạch triển khai}
\begin{frame}{Kế hoạch triển khai}
\begin{tabular}{ |c|c|l| }
    \multicolumn{3}{}{} \\ \hline
    \# & Tuần & Nội dung công việc \\ \hline
    1 & 1 - 2 &\vtop{\hbox{\strut Bài báo liên quan mới nhất và bổ sung cơ sở lý thuyết}\hbox{\strut về các kỹ thuật ước lượng số lượng trên dòng dữ liệu}}\\
    \hline
    2 & 3 - 4 &\vtop{\hbox{\strut Thu thập dữ liệu, chuẩn hoá và tiền xử lý. Hiện thực}\hbox{\strut bài toán 1 ước lượng số lượng phần tử trên dòng dữ liệu}}\\
    \hline
    3 & 5 - 6 &\vtop{\hbox{\strut Mở rộng để ước lượng số lượng phần tử trên }\hbox{\strut nhiều dòng dữ liệu. Đánh giá hiệu suất và độ chính xác.}}\\
    \hline
    4 & 7 - 8 &\vtop{\hbox{\strut Phân tích và so sánh kết quả, đánh giá ưu nhược điểm.}\hbox{\strut Đề suất phương pháp tối ưu hiệu suất và độ chính xác.}}\\
    \hline
    5 & 9 - 10 &\vtop{\hbox{\strut Ứng dụng kết quả nghiên cứu.}\hbox{\strut Đề suất hướng phát triển và nghiên cứu tiếp theo.}}\\
    \hline
    6 & 11 - 12 &\vtop{\hbox{\strut Đề xuất và đánh giá các giải pháp}}\\
    \hline
    7 & 1 - 14 & Tổng hợp kết quả và viết báo cáo \\
    \hline
\end{tabular}
\end{frame}
\section{Nội dung dự kiến của luận văn}
\begin{frame}{Nội dung dự kiến của luận văn}
  \textbf{Chương 1: Giới thiệu}. Tầm quan trọng của việc phát triển kỹ thuật đếm 
  số phần tử trên dòng dữ liệu trong ngữ cảnh 
  dữ liệu lớn.\\
  \textbf{Chương 2: Các công trình nghiên cứu liên quan.} Các công trình nghiên cứu 
  liên quan, phương pháp giải quyết vấn đề. Đánh giá tính khả thi của đề tài.\\
  \textbf{Chương 3: Kiến thức nền tảng.} Giới thiệu về tính chất, phương pháp
  truy vấn và xử lý trên dòng dữ liệu. Giới thiệu về HyperLogLog và nguyên lý 
  hoạt động và đánh giá hiệu suất, độ chính xác trên dòng dữ liệu.\\
  \textbf{Chương 4: Hiện thực và thử nghiệm. } Trong chương này sẽ trình bày chi tiết 
  cách thức hiện thực của từng thuật toán.\\
  \textbf{Chương 5: Kết quả và đánh giá.} Trong chương này sẽ nêu ra các kết quả 
  đạt được của các kỹ thuật, cũng như phương pháp đánh giá dựa trên kết quả thực nghiệm.\\
  \textbf{Chương 6: Kết luận.} Đánh giá ưu điểm và nhược điểm của mô hình và 
  đề xuất hướng nghiên cứu phát triển kỹ thuật đếm số phần tử trong tương lai.
\end{frame}

\section{Kết luận}
\begin{frame}{Kết luận}
  Việc giám sát và quản lý số lượng người dùng đóng vai trò quan trọng trong việc tối ưu hóa hiệu quả hoạt động, 
nâng cao trải nghiệm người dùng, hỗ trợ ra quyết định kinh doanh sáng suốt và đảm bảo an ninh mạng cho doanh nghiệp.\\

Phân bổ tài nguyên hợp lý: Đảm bảo hệ thống hoạt động ổn định, tránh quá tải, lãng phí tài nguyên, tối ưu hóa chi phí vận hành.\\

Nâng cao trải nghiệm người dùng: Giảm thiểu lỗi hệ thống, lag, giật, loading lâu, mang đến trải nghiệm mượt mà, thu hút và giữ chân khách hàng.\\

Phát hiện và khắc phục sự cố kịp thời: Nhận diện sớm các dấu hiệu bất thường, sự cố hệ thống, từ đó có biện pháp khắc phục nhanh chóng, hạn chế ảnh hưởng đến hoạt động kinh doanh.\\

Hiểu rõ hành vi người dùng: Phân tích dữ liệu truy cập, hành vi click chuột, sở thích, nhu cầu của người dùng để cá nhân hóa trải nghiệm, đề xuất sản phẩm/dịch vụ phù hợp, 
nâng cao hiệu quả marketing và dự báo xu hướng thị trường.\\

Hỗ trợ ra quyết định kinh doanh: Đánh giá hiệu quả chiến dịch marketing, điều chỉnh chiến lược phù hợp, dự báo nhu cầu thị trường, 
lập kế hoạch sản xuất, kinh doanh hiệu quả, tối ưu hóa nguồn lực, nâng cao lợi nhuận cho doanh nghiệp.\\

Sự bùng nổ của kỷ nguyên dữ liệu lớn đặt ra nhu cầu cấp thiết cho các kỹ thuật xử lý dữ liệu hiệu quả, chính xác. 
Đề tài "NGHIÊN CỨU PHÁT TRIỂN KỸ THUẬT ĐẾM SỐ PHẦN TỬ TRÊN DÒNG DỮ LIỆU" là một đóng góp quan trọng cho lĩnh vực khoa học máy tính. 
Các thuật toán mới được phát triển trong đề tài này giúp nâng cao hiệu quả và độ chính xác của việc xử lý dữ liệu, mang đến những đóng góp 
to lớn cho lĩnh vực khoa học máy tính và mở ra cánh cửa cho vô số ứng dụng thực tiễn.\\
\end{frame}

% {%
% \setbeamertemplate{frame footer}{My custom footer}
% \begin{frame}[fragile]{Frame footer}
%     \themename defines a custom beamer template to add a text to the footer. It can be set via
%     \begin{verbatim}\setbeamertemplate{frame footer}{My custom footer}\end{verbatim}
% \end{frame}
% }

% \begin{frame}{References}
%   Some references to showcase [allowframebreaks] \cite{durand2003loglog, flajolet2007hyperloglog, chabchoub2010sliding, ertl2024exaloglog, heule2013hyperloglog}
% \end{frame}

% \appendix

% \begin{frame}[fragile]{Backup slides}
%   Sometimes, it is useful to add slides at the end of your presentation to
%   refer to during audience questions.

%   The best way to do this is to include the \verb|appendixnumberbeamer|
%   package in your preamble and call \verb|\appendix| before your backup slides.

%   \themename will automatically turn off slide numbering and progress bars for
%   slides in the appendix.
% \end{frame}

\begin{frame}[allowframebreaks]{Tài liệu tham khảo}
  \bibliography{ref}
  \bibliographystyle{unsrt}
\end{frame}

\end{document}
