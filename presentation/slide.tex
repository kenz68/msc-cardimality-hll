\documentclass[10pt]{beamer}
\usepackage[linesnumbered,algoruled,boxed,lined]{algorithm2e}
\usetheme{metropolis}
\usepackage{appendixnumberbeamer}
\usepackage[utf8]{vietnam}
\usepackage{booktabs}
\usepackage[scale=2]{ccicons}

\usepackage{pgfplots}
\usepgfplotslibrary{dateplot}

\usepackage{xspace}
\newcommand{\themename}{\textbf{\textsc{metropolis}}\xspace}
\newcommand{\SubItem}[1]{
    {\setlength\itemindent{15pt} \item[-] #1}
}
\usepackage{listings}
\usepackage{lstautogobble} % Provides autogobble, which is useful to remove indentation based on first line

\lstset{ %
  language=bash,
%  numbers=left,
  numberstyle=\tiny,
  stepnumber=1,
  numbersep=5pt,
  breaklines=true,
  autogobble=true, % Removes indentation based on first line
}

\title{Đề cương luận văn thạc sĩ}
\subtitle{Nghiên cứu phát triển kỹ thuật đếm số phần tử \\ \hspace{2.4cm} trên dòng dữ liệu}
\date{Người hướng dẫn khoa học: \hspace{0.5cm}PGS. TS. THOẠI NAM}
\author{Học viên: Lê Anh Quốc \hspace{1cm} ID: 2070428}
%\institute{\today}
\titlegraphic{\hfill\includegraphics[height=1.5cm]{hcmut.png}}

\begin{document}

\maketitle

\begin{frame}{Outline}
  \setbeamertemplate{section in toc}[sections numbered]
  \tableofcontents[hideallsubsections]
\end{frame}

\section{Giới thiệu}

\begin{frame}[fragile]{Giới thiệu}

\begin{itemize}
  \item Ứng dụng và dịch vụ trực tuyến đóng vai trò quan trọng trong cuộc sống hiện đại.
  \item DAU (\textbf{Daily Active Users}) là chỉ số quan trọng để đánh giá hiệu quả hoạt động của các ứng dụng và dịch vụ này.
  \item Theo dõi DAU giúp:
  \SubItem{Đánh giá mức độ tương tác và quan tâm của người dùng.}
  \SubItem{Đo lường hiệu quả của chiến dịch marketing và quảng cáo.}
  \SubItem{Hỗ trợ ra quyết định kinh doanh.}
\end{itemize}
\end{frame}
\begin{frame}[fragile]{Giới thiệu}

\begin{itemize}
  \item Thách thức:
  \SubItem{Đếm DAU trên dữ liệu lớn và tốc độ truy cập cao.}
  \SubItem{Tổng hợp DAU từ nhiều nguồn dữ liệu khác nhau.}
  \SubItem{Đếm DAU theo nhiều khoảng thời gian và tiêu chí khác nhau.}
  \item Giải pháp:
  \SubItem{Sử dụng các thuật toán đếm hiệu suất cao và tin cậy.}
  \SubItem{Xây dựng hệ thống tổng hợp dữ liệu linh hoạt và đồng bộ.}
\end{itemize}
\end{frame}

\section{Các công trình nghiên cứu liên quan}
\begin{frame}{Các công trình nghiên cứu liên quan}
\begin{itemize}
	\item Philippe Flajolet, Éric Fusy, Olivier Gandouet, and Frédéric Meunier\\ LogLog Counting of Large Cardinalities, 2003 \cite{durand2003loglog}: 
	\SubItem{Thuật toán ước lượng số lượng phần tử với độ chính xác cao và sử dụng ít bộ nhớ.}
	\item Philippe Flajolet, Éric Fusy, Olivier Gandouet, and Frédéric Meunier\\ HyperLogLog: The Analysis of a Near-Optimal Cardinality Estimation Algorithm, 2007 \cite{flajolet2007hyperloglog}: 
	\SubItem{là một cải tiến từ LogLog, thuật toán này có khả năng ước lượng số lượng phần tử lớn hơn $10^9$ với độ chính xác khoảng 2\% chỉ dụng 1.5 kilobytes bộ nhớ, đồng thời có khả năng song song hoá tối ưu và thích nghi với mô hình cửa sổ trượt (sliding window).}
	\item Stefan Heule, Marc Nunkesser, Alexander Hall\\ HyperLogLog in Practice: Algorithmic Improvements for Practical Cardinality Estimation Deployments, 2017 \cite{heule2013hyperloglog}: 
	\SubItem{Phiên bản nâng cấp của HyperLogLog với độ chính xác cao hơn và yêu cầu bộ nhớ ít hơn.}
\end{itemize}
\end{frame}
\begin{frame}{Các công trình nghiên cứu liên quan}
\begin{itemize}
	\item Chabchoub, Yousra and He{\'e}brail, Georges, Sliding hyperloglog: Estimating cardinality in a data stream over a sliding window, 2010 \cite{chabchoub2010sliding}: 
	\SubItem{Cung cấp ước lượng chính xác số lượng phần tử duy nhất trong dòng dữ liệu liên tục theo cơ chế cửa sổ trượt, sử dụng bộ nhớ hiệu quả. Nó đặc biệt hữu ích cho giám sát thời gian thực và phân tích dữ liệu liên tục.}
	\item Ertl, Otmar, ExaLogLog: Space-Efficient and Practical Approximate Distinct Counting up to the Exa-Scale, 2024 \cite{ertl2024exaloglog}: 
	\SubItem{Cấu trúc dữ liệu mới cho việc đếm độc lập xấp xỉ, tương tự như HyperLogLog, nhưng tiêu tốn ít hơn 43\% không gian với cùng lỗi ước lượng.}
\end{itemize}
\end{frame}

\section{Phát biểu bài toán}
\begin{frame}[fragile]{Phát biểu bài toán}
  \begin{itemize}
    \item \textbf{Bài toán 1:} Phát triển thuật toán để ước lượng số lượng phần tử (cardinality estimation) trong một khoảng thời gian trên một dòng dữ liệu (data stream).
    \item \textbf{Bài toán 2:} Mở rộng thuật toán để ước lượng số lượng phần tử trong một khoảng thời gian trên nhiều dòng dữ liệu.
  \end{itemize}
\end{frame}
\section{Mục tiêu, đối tượng và giới hạn nghiên cứu}

\begin{frame}{Mục tiêu}
  \begin{itemize}
      \item Phát triển kỹ thuật đếm số lượng phần tử hiệu quả, có chính xác cao trên dòng dữ liệu.
      \item Nâng cao hiệu suất xử lý dữ liệu lớn, đáp ứng nhu cầu ngày càng tăng trong kỷ nguyên số.
      \item Thực hiện, phân tích kết quả thí nghiệm, rút ra kết luận và đề xuất hướng nghiên cứu tiếp theo.
      \item Đóng góp vào sự phát triển của công nghệ dữ liệu lớn, mở ra tiềm năng ứng dụng rộng lớn trong nhiều lĩnh vực.
  \end{itemize}
\end{frame}

\begin{frame}{Giới hạn, đối tượng nghiên cứu}
  \textbf{Giới hạn}:
  \begin{itemize}
    \item Đề tài tập trung nghiên cứu kỹ thuật đếm số lượng phần tử trên dòng dữ liệu dạng văn bản.
    \item Các kỹ thuật được đề xuất và triển khai có thể chưa áp dụng được cho tất cả các loại dữ liệu.
    \item Nghiên cứu chỉ giới hạn trong thời gian cho phép.
  \end{itemize}
  \textbf{Đối tượng nghiên cứu}:
    \begin{itemize}
        \item Dòng dữ liệu dạng văn bản có chứa nhiều phần tử cần đếm.
        \SubItem{userID, IP address, words, etc}
        \item Các kỹ thuật đếm số lượng phần tử hiện có và mới được đề xuất.
    \end{itemize}
\end{frame}

\section{Cơ sở lý thuyết}

\begin{frame}[fragile]{Cơ sở lý thuyết}
\begin{itemize}
  \item Các thuật toán xác suất cho việc ước lượng trong thực tế là họ của LogLog:
  \SubItem{\textit{LogLog}, 2003}
  \SubItem{\textit{HyperLogLog}, 2007}
  \SubItem{\textit{HyperLogLog++}, 2017}
  \item Sử dụng phương pháp tương tự thuật toán Đếm Xác Suất
  \SubItem{ước lượng số lượng $n$ bằng cách quan sát số lượng lớn nhất của các số không dấu
  trong biểu diễn nhị phân}
  \SubItem{Hàm băm $h \gets \{0,1,...,2^M-1\}$, với độ dài $M$}
\end{itemize}
\[
    h(x) = j = \sum\limits_{k=0}^{M-1}j_k\cdot2^k := \left(i_0i_1...i_{M-1}\right)_2,i_k \in \{0,1\}
\]
\end{frame}
\begin{frame}[fragile]{Cơ sở lý thuyết}
  \begin{itemize}
    \item Bằng cách sử dụng một phần của giá trị băm $h(x)$ để chia tập dữ liệu ban
    đầu thành một số tập con.
    \item Phần còn lại được sử dụng để cập nhật bộ đếm dựa trên việc quan sát mẫu $0^k1$
    \item Hạng tương đương với vị trí bit 1 đầu tiên từ trái qua phải (rank)
  \end{itemize}

\[
    rank(i)=\left\{
                \begin{array}{ll}
                    \min\limits_{i_k\neq 0}, \indent\text{for } i > 0,\\
                    M\indent\indent\text{for } i = 0
                \end{array}
            \right.
\]

\begin{itemize}
  \item Tiết kiệm dung lượng lưu trữ và có độ chính xác cao.
\end{itemize}
\end{frame}

\begin{frame}{LogLog algorithm}
\begin{itemize}
	\item Ý tưởng cơ bản: tính \textbf{rank(h(x))}, \textbf{h} là hash func, \textbf{x} input.
	\item Có thể mong đợi rằng khoảng $\frac{n}{2^k}$ phần tử có thể có $rank(\cdot) = k$, $n$ là tổng số phần tử được lập chỉ mục vào một bộ đếm, hạng tối đa:
  \[R = \underset{x \in D}{\max}\left(rank(x)\right) \approx log_2n.\]
  \item Tuy nhiên, ước lượng như vậy có sai số khoảng $\pm1.87$ lần nhị phân, điều này 
  không thực tế. 
  \item Để giảm sai số, \textit{LogLog} chia tập dữ liệu thành $m = 2^p$ tập con $S_0, S_1,..., S_{m-1}$. Với mỗi giá trị băm $h(x)$:
	\SubItem{$p$ bit đầu để tính chỉ số $j = \left(i_0i_1...i_{p-1}\right)_2$} 
  	\SubItem{(m-p) bit được lập chỉ mục vào bộ đếm tương ứng COUNTER[j] để tính hạng và quan sát $R_j$.}

\end{itemize}

\end{frame}
\begin{frame}{LogLog algorithm}
\begin{itemize}
\item Dưới sự phân phối công bằng, mỗi tập con nhận $\frac{n}{m}$ phần tử, do đó quan sát $R_j$ từ các bộ đếm $\{COUNTER[j]\}_{j=0}^{m-1}$ có thể cung cấp 
  một dấu hiệu về giá trị của $\log_2{\frac{n}{m}}$, và bằng cách sử dụng trung bình 
  số học của chúng với một số sự hiệu chỉnh, chúng ta có thể giảm thiểu phương sai 
  của một quan sát duy nhất: 
  \[n = \alpha_m \cdot m \cdot 2 ^{\frac{1}{m}\sum\limits_{j=0}^{m-1}R_j}\]
  với $\alpha_m = \left(\Gamma\left(-\frac{1}{m}\right)\cdot \frac{1-2\frac{1}{m}}{\log_2}\right)^m$, $\Gamma(\cdot)$ là gamma function.\\ 
  \item Tuy nhiên, đối với hầu hết các trường hợp thực tế, $m \geq 64$ là đủ để chỉ sử dụng 
  $\alpha_m \approx 0.39701$.
\end{itemize}
\end{frame}
\begin{frame}{LogLog algorithm}
  \begin{algorithm}[H]
    \vspace{0.25cm}
    \DontPrintSemicolon
    \LinesNumberedHidden
    \caption[]{Estimatin cardinality with \textit{LogLog}}
    \KwIn{Dataset D}
    \KwIn{Array of $m$ \textit{LogLog} counters with hash function $h$}
    \KwOut{Cardinality estimation}
    $COUNTER[j] \gets $0, $j = 0...m - 1$\\
    \For{$x \in D$}{
        $i \gets $h(x) := $(i_0i_1...i_{M-1})_2$, $i_k \in \{$0,1\}\\
        $j \gets $($i_0i_1...i_{M-1})_2$\\
        $r \gets $rank($(i_pi_{p+1}...i_{M-1})_2$)\\
        $COUNTER[j] \gets \max($COUNTER[j],r)
    }
    $R \gets \frac{1}{m} \sum\limits_{k=0}^{m-1}$COUNTER[j] \\
    \Return{$\alpha_m \cdot m \cdot 2^R$}
    \vspace{0.25cm}
\end{algorithm}
\end{frame}

\begin{frame}{LogLog algorithm}
Sai số tiêu chuẩn $\delta$ của thuật toán \textit{LogLog}:
\[\delta \approx \frac{1.3}{\sqrt{m}}\]
\begin{itemize}
  \item $m = 256$, $\delta \approx 8\%$
  \item $m = 1024$, nó giảm xuống còn khoảng 4\%.
  \item Yêu cầu lưu trữ có thể được ước tính là $O(log_2log_2n)$ bits nếu cần đến đếm đến \textbf{n}.
  \item So sánh với thuật toán Đếm Xác suất trong đó mỗi bộ đếm yêu cầu 16 hoặc 32 bit, thuật toán \textit{LogLog} yêu cầu bộ đếm nhỏ hơn 
nhiều $\{COUNTER[j]\}_{j=0}^{m-1}$, thường là 5 bit mỗi bộ đếm. 
\item Tuy nhiên, trong khi thuật toán \textit{LogLog} cung cấp hiệu quả lưu trữ tốt hơn 
so với thuật toán Đếm Xác suất, nó đôi chút ít chính xác hơn.
\end{itemize}

\end{frame}

\begin{frame}{LogLog algorithm}
\begin{itemize}
  \item Giả sử chúng ta cần đếm định lượng cho đến $2^{30}$, tức là khoảng 1 tỷ, với độ chính xác khoảng 4\%. Như đã đề cập, cho sai số tiêu chuẩn như vậy, 
cần $m = 1024$ ngăn, mỗi ngăn sẽ nhận xấp xỉ $\frac{n}{m} = 2^{20}$ phần tử.
$log_2\left(log_{2}2^{20}\right) \approx 4.32$, do đó, chỉ cần phân bổ khoảng 5 bit cho mỗi ngăn (tức là một giá trị nhỏ hơn 32).
	\item Do đó, để ước lượng định lượng lên đến khoảng $10^9$ với sai số tiêu chuẩn là 4\%, thuật toán yêu cầu 1024 ngăn với 5 bit mỗi ngăn, 
tức là tổng cộng 640 byte.
\end{itemize}

\end{frame}

\begin{frame}{HyperLogLog algorithm}
\begin{itemize}
	\item Đã được đề xuất bởi Philippe Flajolet, Eric Fusy, Olivier Gandouet và 
  Frederic Meunier vào năm 2007 \cite{flajolet2007hyperloglog}.
  \item Sử dụng hàm băm 32-bit và hàm đánh giá có các sửa lỗi \textbf{bias} khác nhau.
  \item Xử lý các định lượng lên đến $10^9$ với một hàm băm 32-bit đơn lẻ $h$ chia tập dữ liệu thành $m = 2^p$ tập con, với $p \in 4...16$.
  \item Sử dụng trung bình điều hoà (\textbf{hamonic} mean) thay vì sử dụng trung bình hình học (geometric mean) như phiên bản gốc LogLog

\end{itemize}
\[
      \hat{n} \approx \alpha_m \cdot m^2 \cdot \left(\sum_{j=0}^{m-1}2^{-COUNTER[j]}\right),
  \]
  \indent với
  \[
      \alpha_m = \left(m\int_0^\infty\left(\log_2\left(\frac{2+x}{1+x}\right)\right)^m\text{dx}\right)^{-1}.    
  \]
\end{frame}
\begin{frame}{HyperLogLog algorithm}
  Ý tưởng đằng sau việc sử dụng trung bình điều hòa là nó giảm phương sai do tính chất của nó để kiểm soát các phân phối xác suất lệch.\\
  \begin{center}
      \textbf{Table 1:} $\alpha_m$ for most used values of $m$\\
      \begin{tabular}{ |c|c| }
          \multicolumn{2}{}{} \\ \hline
          m & $\alpha_m$ \\ \hline
          64 &  0.673 \\ \hline
          256 & 0.697 \\ \hline
          1024 & 0.709 \\ \hline
          $\ge 2^7$ & $\frac{0.7213 \cdot m}{m + 1.079}$ \\ \hline
      \end{tabular}
  \end{center}
\end{frame}
\begin{frame}{HyperLogLog algorithm}
  Tuy nhiên, ước lượng $\hat{\textbf{n}}$ yêu cầu một sự điều chỉnh cho các phạm vi nhỏ và lớn do lỗi phi tuyến. Flajolet và các đồng nghiệp 
  đã tìm thấy từ kinh nghiệm rằng đối với các định lượng nhỏ $n < \frac{5}{2}m$ để đạt được ước lượng tốt hơn, thuật toán \textit{HyperLogLog} 
  có thể được sửa lỗi bằng Đếm Tuyến Tính bằng cách sử dụng một số bộ đếm COUNTER[j] khác không (nếu một bộ đếm có giá trị là không, 
  chúng ta có thể nói chắc chắn rằng tập con cụ thể đó là trống).\\
  
  Do đó, cho các phạm vi định lượng khác nhau, được biểu diễn dưới dạng các khoảng trên ước lượng $\hat{n}$, 
  thuật toán cung cấp các sửa lỗi sau:
  \[
      n = \left\{ \begin{array}{lcl}
          \text{LINEARCOUNTER,} & \hat{n} \leq \frac{5}{2}m \mbox{ and } \exists_j : COUNTER[j] \neq 0 \\ 
          -2^{32}\log\left(1-\frac{\hat{n}}{2^{32}}\right), & \hat{n}>\frac{1}{30}2^{32} \\
          \hat{n}, & \textit{otherwise.}
          \end{array}\right.    
  \]
\end{frame}
\begin{frame}{HyperLogLog algorithm}
  Tuy nhiên, đối với $n = 0$, sự sửa lỗi có vẻ không đủ và thuật toán luôn trả về một kết quả xấp xỉ $0.7m$.\\
  Vì thuật toán \textit{HyperLogLog} sử dụng một hàm băm 32-bit, khi định lượng tiến gần đến $2^{32}\approx 4 \cdot 10^9$, 
  hàm băm gần như đạt đến giới hạn của nó và xác suất va chạm tăng lên. Đối với các phạm vi lớn như vậy, thuật toán \textit{HyperLogLog} 
  ước lượng số lượng giá trị băm khác nhau và sử dụng nó để xấp xỉ định lượng. Tuy nhiên, trong thực tế, có nguy cơ rằng một số lượng cao hơn 
  không thể được đại diện và sẽ bị mất, ảnh hưởng đến độ chính xác.\\
  
  \begin{quote}
      Xem xét một hàm băm mà ánh xạ vũ trụ thành các giá trị có độ dài M bit. Tối đa, một hàm như vậy có thể mã hóa $2^M$ 
      giá trị khác nhau và nếu định lượng ước lượng $n$ tiến dần đến giới hạn này, đụng độ hàm băm trở nên ngày càng có khả năng xảy ra.
      \vspace{0.25cm}
  \end{quote}
  \newpage
  \indent Không có bằng chứng cho thấy một số hàm băm phổ biến (ví dụ: MurmurHash3, MD5, SHA-1, SHA-256) 
  hoạt động đáng kể tốt hơn các hàm khác trong các thuật toán \textit{HyperLogLog} hoặc các biến thể của nó.
\end{frame}
\begin{frame}{HyperLogLog algorithm (1/2)}
  \begin{algorithm}[H]
    \DontPrintSemicolon
%    \LinesNumberedHidden
    \caption[]{Estimatin cardinality with \textit{HyperLogLog}}
    \KwIn{Dataset D}
    \KwIn{Array of $m$ \textit{LogLog} counters with hash function $h$}
    \KwOut{Cardinality estimation}
    $COUNTER[j] \gets $0, $j = 0...m - 1$\\
    \For{$x \in D$}{
        $i \gets $h(x) := $(i_0i_1...i_{M-1})_2$, $i_k \in \{$0,1\}\\
        $j \gets $($i_0i_1...i_{M-1})_2$\\
        $r \gets $rank($(i_pi_{p+1}...i_{M-1})_2$)\\
        $COUNTER[j] \gets \max($COUNTER[j],r)
    }
    $R \gets \frac{1}{m} \sum\limits_{k=0}^{m-1}$COUNTER[j] \\
    $\hat{n} = \alpha_m \cdot m^2 \cdot \frac{1}{R}$ \\
    $n \gets \hat{n}$ \\

\end{algorithm}
\end{frame}
\begin{frame}{HyperLogLog algorithm (2/2)}
  \begin{algorithm}[H]
    \DontPrintSemicolon
    \LinesNumberedHidden

    $R \gets \frac{1}{m} \sum\limits_{k=0}^{m-1}$COUNTER[j] \\
    $\hat{n} = \alpha_m \cdot m^2 \cdot \frac{1}{R}$ \\
    $n \gets \hat{n}$ \\
    \If{$\hat{n} \le \frac{5}{2}m$}
    {
        $Z \gets \underset{j=0...m-1}{count}\left(COUNTER[j] == 0\right)$\\
        \If{$Z \ne 0$}{$n \gets m \cdot \log\left(\frac{m}{Z}\right)$}
    }
    \ElseIf{$\hat{n} > \frac{1}{30}2^{32}$}
    {
        $n \gets -2^{32}\cdot\log\left(1-\frac{n}{2^{32}}\right)$
    }
    \Return{$n$}
\end{algorithm}
\end{frame}
\begin{frame}{HyperLogLog algorithm}
\begin{itemize}
	\item Tương tự LogLog, sai số tiêu chuẩn $\delta$: \[\delta \approx \frac{1.04}{\sqrt{m}}.\]
	\item Yêu cầu bộ nhớ không tăng tuyến tính theo số lượng phần tử, phân bổ $(M=p$) bit cho các giá trị băm và có tổng cộng $m = 2^p$ bộ đếm, bộ nhớ cần thiết là:
  \[\lceil\log_2\left(M + 1 - p\right)\rceil\cdot2^p \text{ bits }\]
  \item Sử dụng hàm băm 32-bit và độ chính xác $p \in 4...16$, yêu cầu bộ nhớ là $5\cdot 2^p$ bit. Do đó, thuật toán \textit{HyperLogLog} cho phép ước lượng các định lượng vượt xa $10^9$ với độ chính xác thông thường là $2\%$ trong khi chỉ sử dụng một bộ nhớ chỉ \textbf{1.5 KB}.
  \item Ví dụ, \textbf{Redis} duy trì cấu trúc dữ liệu \textit{HyperLogLog} của \textbf{12 KB} để xấp xỉ các định lượng $\delta \approx 0.81.\%$.
\end{itemize}
\end{frame}

\begin{frame}{HyperLogLog: các thư viện và công cụ nổi bật }
  \begin{itemize}
    \item \textbf{Apache DataSketches:} một thư viện Java cung cấp các thuật toán xác suất và thống kê. Được sử dụng rộng rãi trong các hệ thống Big Data như 
    Apache Druid, Apache Kafka và Apache Hive.
    \item \textbf{Redis:} hệ thống lưu trữ dữ liệu dạng key-value phổ biến, cung cấp 
    cấu trúc dữ liệu HyperLogLog tích hợp sẵn. Sử dụng các lệnh như PFADD, PFCOUNT 
    và PFMERGE để làm việc với HLL.
    \item \textbf{Google BigQuery:} sử dụng HyperLogLog++ cho các chức năng thống kê và phân tích dữ liệu lớn.
    \item \textbf{PostgreSQL:} Extension postgreSQL-hll giúp thực hiện các truy vấn với số lượng phần tử duy nhất một cách hiệu quả.
    \item \textbf{Amazon Redshift:} hỗ trợ HyperLogLog để tối ưu hóa các truy vấn thống kê và giảm thiểu dung lượng bộ nhớ cần thiết.
    \item \textbf{Apache Flink:} một nền tảng xử lý luồng dữ liệu phân tán, 
    có tích hợp HyperLogLog để xử lý các phép tính phức tạp trên dữ liệu luồng.
  \end{itemize}
\end{frame}

\begin{frame}{Ví dụ: Redis (1/3)}
Redis là một hệ thống lưu trữ dữ liệu dạng key-value phổ biến và hỗ trợ nhiều cấu trúc dữ liệu mạnh mẽ, trong đó có HyperLogLog. Redis cung cấp các lệnh chuyên biệt để làm việc với HyperLogLog, bao gồm PFADD, PFCOUNT và PFMERGE.\\
\textbf{Các lệnh cơ bản của HyperLogLog trong Redis:}
  \begin{itemize}
    \item \textbf{PFADD:} Thêm các phần tử vào HyperLogLog.
    \item \textbf{PFCOUNT:} Ước lượng số phần tử duy nhất trong HyperLogLog.
    \item \textbf{PFMERGE:} Hợp nhất nhiều HyperLogLog thành một.
  \end{itemize}
\end{frame}

\begin{frame}{Ví dụ: Redis (2/3)}
Giả sử chúng ta có một ứng dụng web và muốn theo dõi số lượng người dùng duy nhất truy cập vào website mỗi ngày.\\
\textbf{Các lệnh cơ bản của HyperLogLog trong Redis:}
  \begin{itemize}
    \item \textbf{Cài đặt Redis:} 
   	\SubItem{sudo apt-get update}
   	\SubItem{sudo apt-get install redis-server}
    \item \textbf{Khởi động Redis server:} 
    \SubItem{redis-server}   
    \item \textbf{Kết nối tới Redis:}
    \SubItem{redis-cli}
   \end{itemize}
\end{frame}

\begin{frame}[fragile]{Ví dụ: Redis (3/3)}
\begin{itemize}
	\item \textbf{Thêm người dùng}:
	\begin{lstlisting}
PFADD unique_visitors:2024-05-25T08 user1 user2 user3
PFADD unique_visitors:2024-05-25T08 user2 user4
PFADD unique_visitors:2024-05-25T09 user5 user6
PFADD unique_visitors:2024-05-25T10 user1 user7
\end{lstlisting}
\item \textbf{Ước lượng số người dùng trong giờ 08:00 ngày 2024-05-25}:
	\begin{lstlisting}
PFCOUNT unique_visitors:2024-05-25T08
\end{lstlisting}
\item \textbf{Giả sử chúng ta muốn hợp nhất dữ liệu người dùng duy nhất từ ba giờ khác nhau (08:00, 09:00, và 10:00).}:
	\begin{lstlisting}
PFMERGE unique_visitors:2024-05-25:morning unique_visitors:2024-05-25T08 unique_visitors:2024-05-25T09 unique_visitors:2024-05-25T10
PFCOUNT unique_visitors:2024-05-25:morning
\end{lstlisting}
\end{itemize}
\end{frame}

\begin{frame}[fragile]{Cách HyperLogLog lưu trữ dữ liệu trong Redis}
Redis sử dụng "sparse representation" cho các bộ đếm nhỏ và "dense representation" cho các bộ đếm lớn hơn.\\
Dung lượng bộ nhớ phụ thuộc vào số lượng thanh ghi (registers), 
và mỗi thanh ghi lưu trữ thông tin về vị trí của bit đầu tiên là 1 
trong chuỗi băm.\\
\vspace{0.2cm}
\textbf{Số lượng thanh ghi (m)}:
\begin{itemize}
	\item $m = 2^p$, trong đó p là số bit để xác định số thanh ghi.
	\item Giá trị mặc định p trong Redis là \textbf{14} => có $2^{14} = 16384$ thanh ghi.
\end{itemize}
\textbf{Dung lượng bộ nhớ cần thiết}
\begin{itemize}
	\item Mỗi thanh ghi cần \textbf{6 bit} để lưu trữ vị trí của bit đầu tiên là 1.
	\item Tổng dung lượng bộ nhớ cần thiết cho HyperLogLog trong Redis có thể 
  tính theo công thức (với m = 16384 thanh ghi):\\ 
	Memory = 16384 x 6 bits = 98304 bits = 12288 bytes = \textbf{12 KB}
\end{itemize}
\end{frame}

\begin{frame}[fragile]{Tính toán dung lượng cần thiết cho nhiều tập hợp HLL}
Nếu bạn muốn lưu trữ nhiều tập hợp HyperLogLog trong Redis, ví dụ như 
theo dõi số người dùng duy nhất theo giờ, cần nhân dung lượng bộ nhớ 
của một tập hợp HLL với số lượng tập hợp bạn có.\\
Giả sử bạn muốn lưu trữ dữ liệu người dùng duy nhất cho mỗi giờ trong 
một ngày (24 giờ):\\
\begin{center}
  Total Memory = 12 KB x 24 = \textbf{288 KB}
\end{center} 
Nếu bạn muốn lưu trữ dữ liệu theo từng giờ cho nhiều ngày, bạn chỉ cần 
nhân thêm số lượng ngày:
\begin{center}
  Total Memory for 30 days = 288 KB/day × 30 = 8640 KB = \textbf{8.64 MB}
\end{center} 
\end{frame}

\begin{frame}[fragile]{Tối ưu hóa dung lượng bộ nhớ}
  \begin{itemize}
    \item Redis sử dụng "sparse representation" cho các bộ đếm nhỏ hơn, 
    giúp tiết kiệm bộ nhớ khi số lượng phần tử trong HyperLogLog còn ít.
    \item Khi số lượng phần tử tăng, Redis chuyển sang "dense representation" 
    để đảm bảo độ chính xác và hiệu suất.
    \item Redis tự động chuyển đổi giữa hai biểu diễn này dựa trên số lượng 
    phần tử và mức độ lấp đầy của các thanh ghi, giúp tối ưu hóa việc 
    sử dụng bộ nhớ và đảm bảo độ chính xác cao trong ước lượng số lượng 
    phần tử duy nhất.
  \end{itemize}
\end{frame}

\begin{frame}[fragile]{Sparse Representation}
  \textbf{Đặc điểm:}
  \begin{itemize}
    \item \textbf{Tiết kiệm bộ nhớ}: Sparse Representation được thiết kế để 
    tiết kiệm bộ nhớ khi số lượng phần tử trong tập hợp còn ít.
    \item \textbf{Cấu trúc nén}: Lưu trữ các cặp (index, value) để chỉ 
    lưu trữ thông tin cần thiết về các thanh ghi được cập nhật.
    \item \textbf{Hiệu quả cho các tập hợp nhỏ}: Rất hiệu quả khi số lượng 
    phần tử còn ít, vì không cần lưu trữ toàn bộ 16384 thanh ghi.
  \end{itemize}
  \textbf{Ước lượng số phần tử:}
  \begin{itemize}
    \item \textbf{Cơ chế hoạt động}: Các giá trị băm của phần tử được ánh xạ 
    tới một trong 16384 thanh ghi, nhưng chỉ các thanh ghi có giá trị 
    khác 0 mới được lưu trữ.
    \item \textbf{Độ chính xác}: Độ chính xác của ước lượng trong sparse representation tương tự như trong dense representation khi số lượng phần tử còn nhỏ, vì các thanh ghi được quản lý chặt chẽ và thông tin được lưu trữ một cách nén.
  \end{itemize}
\end{frame}

\begin{frame}[fragile]{Dense Representation}
  \textbf{Đặc điểm:}
  \begin{itemize}
    \item \textbf{Sử dụng bộ nhớ cố định}:  Khi số lượng phần tử lớn, 
    HyperLogLog chuyển sang Dense Representation, lưu trữ toàn bộ mảng 16384 
    thanh ghi với mỗi thanh ghi sử dụng 6 bit.
    \item \textbf{Hiệu quả cho các tập hợp lớn}: Dense Representation trở nên 
    hiệu quả hơn khi số lượng phần tử tăng, vì việc nén không còn mang lại 
    lợi ích về bộ nhớ so với việc lưu trữ toàn bộ thanh ghi.
  \end{itemize}
  \textbf{Ước lượng số phần tử:}
  \begin{itemize}
    \item \textbf{Cơ chế hoạt động}: Tương tự như sparse representation, 
    nhưng toàn bộ mảng thanh ghi được lưu trữ và sử dụng để tính toán 
    ước lượng.
    \item \textbf{Độ chính xác}: Độ chính xác của ước lượng trong 
    dense representation cao hơn khi số lượng phần tử lớn, vì nó có thể 
    quản lý thông tin của tất cả các thanh ghi mà không cần nén.
  \end{itemize}
\end{frame}

\section{Phương pháp thực hiện}
\begin{frame}[fragile]{Bài toán 1}
Phát triển thuật toán để ước lượng số lượng phần tử (cardinality estimation) trong 
một khoảng thời gian trên một dòng dữ liệu (data stream): 

\begin{itemize}
  \item Bước 1: Xác định khoảng thời gian
  Đầu tiên, chúng tôi sẽ xác định khoảng thời gian mà chúng tôi muốn đếm số lượng 
  phần tử. Ví dụ, mỗi giờ hoặc mỗi phút.
  \item Bước 2: Lưu trữ HyperLogLog
  Tiếp theo, chúng tôi sẽ lưu trữ cấu trúc HyperLogLog cho mỗi khoảng thời gian. 
  Cấu trúc dữ liệu sẽ bao gồm cặp $\left< T, HLL_1\right>$, trong đó $T$ là 
  thời điểm đại diện cho khung thời gian cụ thể.
  \item Bước 3: Sử dụng kết quả
  Cuối cùng, khi cần, chúng tôi có thể truy vấn và sử dụng kết quả từ các cấu trúc 
  HyperLogLog lưu trữ theo khung thời gian để ước lượng số lượng phần tử trong 
  mỗi khoảng thời gian.
\end{itemize}
\end{frame}
\begin{frame}[fragile]{Bài toán 2 (1/2)}
  Mở rộng thuật toán để ước lượng số lượng phần tử trong một khoảng thời gian trên nhiều dòng dữ liệu:\\
\vspace{0.1cm}
  \textit{ Ví dụ khi chúng ta cần biết có bao nhiêu người dùng đã đăng nhập vào hệ thống vào ngày hôm qua, do dữ liệu người dùng được lưu ở trên nhiều hệ thống như web, application và cũng như trên các bộ phận khác nhau của doanh nghiệp. Khi đó chúng ta sẽ có nhiều nguồn dữ liệu khác nhau và cần một thuật toán để kết hợp các nguồn dữ liệu này để tổng hợp cho ra ước lượng
  số lượng cuối cùng.}

\end{frame}
\begin{frame}[fragile]{Bài toán 2 (2/2)}
  \begin{itemize}
      \item Bước 1: Lưu trữ dữ liệu
      \SubItem{Lưu trữ dữ liệu theo khung thời gian $\left< T, HLL\right>$}
      \item Bước 2: Tổng hợp HLL từ các dữ liệu từ nhiều nơi khác nhau:\\
      $\left< T, HLL_1\right>, \left< T, HLL_2\right>,...,\left< T, HLL_N\right>$\\
      \SubItem{$T$ là khoảng thời gian cần tổng hợp} \\
      \SubItem{$HLL_1$ là dữ liệu HyperLogLog trong khoảng thời gian}
      \item Bước 3: Ước lượng số phần tử
      \[E = \alpha_m \cdot m^2 \cdot \left( \sum_{j=1}^{m} 2^{-M[j]} \right)^{-1}\]
      \SubItem{$E$ là ước lượng số lượng phần tử duy nhất}
      \SubItem{$\alpha_m$ là hằng số}
      \SubItem{$m$ là số lượng register trong cấu trúc HyperLogLog}
      \SubItem{$M[j]$ là giá trị của register thứ j}
\end{itemize}
\end{frame}
\section{Kế hoạch triển khai}
\begin{frame}{Kế hoạch triển khai}
\begin{tabular}{ |c|c|l| }
    \multicolumn{3}{}{} \\ \hline
    \# & Tuần & Nội dung công việc \\ \hline
    1 & 1 - 2 &\vtop{\hbox{\strut Bài báo liên quan mới nhất và bổ sung cơ sở lý thuyết}\hbox{\strut về các kỹ thuật ước lượng số lượng trên dòng dữ liệu}}\\
    \hline
    2 & 3 - 4 &\vtop{\hbox{\strut Thu thập dữ liệu, chuẩn hoá và tiền xử lý. Hiện thực}\hbox{\strut bài toán 1 ước lượng số lượng phần tử trên dòng dữ liệu}}\\
    \hline
    3 & 5 - 6 &\vtop{\hbox{\strut Mở rộng để ước lượng số lượng phần tử trên }\hbox{\strut nhiều dòng dữ liệu. Đánh giá hiệu suất và độ chính xác.}}\\
    \hline
    4 & 7 - 8 &\vtop{\hbox{\strut Phân tích và so sánh kết quả, đánh giá ưu nhược điểm.}\hbox{\strut Đề suất phương pháp tối ưu hiệu suất và độ chính xác.}}\\
    \hline
    5 & 9 - 10 &\vtop{\hbox{\strut Ứng dụng kết quả nghiên cứu.}\hbox{\strut Đề suất hướng phát triển và nghiên cứu tiếp theo.}}\\
    \hline
    6 & 11 - 12 &\vtop{\hbox{\strut Đề xuất và đánh giá các giải pháp}}\\
    \hline
    7 & 1 - 14 & Tổng hợp kết quả và viết báo cáo \\
    \hline
\end{tabular}
\end{frame}

\begin{frame}{Nội dung dự kiến của luận văn}
  \textbf{Chương 1: Giới thiệu}. Tầm quan trọng của việc phát triển kỹ thuật đếm 
  số phần tử trên dòng dữ liệu trong ngữ cảnh 
  dữ liệu lớn.\\
  \textbf{Chương 2: Các công trình nghiên cứu liên quan.} Các công trình nghiên cứu 
  liên quan, phương pháp giải quyết vấn đề. Đánh giá tính khả thi của đề tài.\\
  \textbf{Chương 3: Kiến thức nền tảng.} Giới thiệu về tính chất, phương pháp
  truy vấn và xử lý trên dòng dữ liệu. Giới thiệu về HyperLogLog và nguyên lý 
  hoạt động và đánh giá hiệu suất, độ chính xác trên dòng dữ liệu.\\
  \textbf{Chương 4: Hiện thực và thử nghiệm. } Trong chương này sẽ trình bày chi tiết 
  cách thức hiện thực của từng thuật toán.\\
  \textbf{Chương 5: Kết quả và đánh giá.} Trong chương này sẽ nêu ra các kết quả 
  đạt được của các kỹ thuật, cũng như phương pháp đánh giá dựa trên kết quả thực nghiệm.\\
  \textbf{Chương 6: Kết luận.} Đánh giá ưu điểm và nhược điểm của mô hình và 
  đề xuất hướng nghiên cứu phát triển kỹ thuật đếm số phần tử trong tương lai.
\end{frame}

% {%
% \setbeamertemplate{frame footer}{My custom footer}
% \begin{frame}[fragile]{Frame footer}
%     \themename defines a custom beamer template to add a text to the footer. It can be set via
%     \begin{verbatim}\setbeamertemplate{frame footer}{My custom footer}\end{verbatim}
% \end{frame}
% }

% \begin{frame}{References}
%   Some references to showcase [allowframebreaks] \cite{durand2003loglog, flajolet2007hyperloglog, chabchoub2010sliding, ertl2024exaloglog, heule2013hyperloglog}
% \end{frame}

% \appendix

% \begin{frame}[fragile]{Backup slides}
%   Sometimes, it is useful to add slides at the end of your presentation to
%   refer to during audience questions.

%   The best way to do this is to include the \verb|appendixnumberbeamer|
%   package in your preamble and call \verb|\appendix| before your backup slides.

%   \themename will automatically turn off slide numbering and progress bars for
%   slides in the appendix.
% \end{frame}

\begin{frame}[allowframebreaks]{Tài liệu tham khảo}
  \bibliography{ref}
  \bibliographystyle{ieeetr}
\end{frame}

\begin{frame}[standout]
  Questions?
\end{frame}

\end{document}
